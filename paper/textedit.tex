\section{Discussion: Merging Text Edits}\label{sec:bad-merge}

The datatypes we have specified in \S~\ref{sec:datatypes} can support a wide range of applications.
For example, the list datatype can be used to implement a collaborative text editor: by treating the text as a list of individual characters, every edit can be expressed as a sequence of insertion or deletion operations on the list.

The problem of collaborative text editing has been studied extensively, using two main approaches: Operational Transformation and CRDTs.
We discuss this prior work in \S~\ref{sec:relwork}.
We will now highlight a scenario that, to our knowledge, has not been considered by any previous work on collaborative text editing.

Consider the execution illustrated in Figure~\ref{fig:bad-merge}.
In this example, two users are concurrently editing a text document that initially reads ``Hello!''.
The user on the left changes it to read ``Hello Alice!'', while concurrently the user on the right changes the document to read ``Hello Charlie!''.
When the concurrent edits are merged, the algorithm randomly interleaves the two insertions of ``~Alice'' and ``~Charlie'' character by character, resulting in an unreadable jumble of characters.

\begin{figure}
\centering
\begin{tikzpicture}
  \tikzstyle{box}=[rectangle,draw,inner xsep=6pt,text height=9pt,text depth=2pt]
  \tikzstyle{every path}=[draw,-{Stealth[length=3.5mm]}]
  \node [box] (start) at (3,4) {\texttt{Hello!}};
  \node [box] (left)  at (0,2) {\texttt{Hello Alice!}};
  \node [box] (right) at (6,2) {\texttt{Hello Charlie!}};
  \node [box] (merge) at (3,0) {\texttt{Hello Al Ciharcliee!}};
  \draw (start) to node [left,inner xsep=10pt,font=\footnotesize]  {Insert ``~Alice'' between ``o'' and ``!''} (left);
  \draw (start) to node [right,inner xsep=10pt,font=\footnotesize] {Insert ``~Charlie'' between ``o'' and ``!''} (right);
  \draw (left)  -- (merge);
  \draw (right) -- (merge);
  \node [text width=3cm,text badly centered,font=\footnotesize] at (3,1) {Merge concurrent edits};
\end{tikzpicture}
\caption{Two concurrent insertions at the same position are interleaved.}\label{fig:bad-merge}
\end{figure}

The problem is even worse if the concurrent insertions are not just a single word, but an entire paragraph or section.
In these cases, interleaving the users' insertions would most likely result in an entirely incomprehensible text that would have to be deleted and rewritten.
Even though the merge in Figure~\ref{fig:bad-merge} is so obviously undesirable, there is to our knowledge no formal specification of collaborative text editing that rules out such an interleaving of insertions.

\begin{theorem}\label{thm:attiya-allows-interleaving}
    The $\mathcal{A}_\textsf{strong}$ specification of collaborative text editing by Attiya et al. \cite{Attiya:2016kh} allows the outcome in Figure~\ref{fig:bad-merge}; that is, an algorithm that interleaves concurrent insertions at the same position may nevertheless satisfy the $\mathcal{A}_\textsf{strong}$ specification.
    Moreover, the text editing CRDT algorithms Logoot \cite{Weiss:2009ht,Weiss:2010hx} and LSEQ \cite{Nedelec:2016eo,Nedelec:2013ky} also allow this outcome.
\end{theorem}
\begin{proof}
    Follows directly from the respective definitions, which are all based on the idea of assigning each character a position in a totally ordered identifier space, such that the order of identifiers corresponds to the order of characters in the document.
    When a new character is inserted, it is assigned an identifier that lies somewhere between the identifiers of its predecessor and successor.
    However, when concurrent insertions with the same predecessor and successor are performed, those insertions are ordered arbitrarily.
    Repeated insertions within the same predecessor-successor interval may thus be interleaved arbitrarily.

    We also performed tests with open source implementations of Logoot \cite{AhmedNacer:2011ke,ReplicationBenchmark} and LSEQ \cite{LSEQTree,Nedelec:2016eo}, and observed this interleaving anomaly occurring in practice.
\end{proof}

Rather than interleaving characters, a better approach to merging is to keep all insertions by a particular user together as a continuous sequence.
With this constraint, there are two acceptable merged results in the example of Figure~\ref{fig:bad-merge}: either ``Hello Alice Charlie!'' or ``Hello Charlie Alice!''.
The choice between these two outcomes is arbitrary, as there is no \emph{a priori} requirement for one user's insertions to come before the other's.

\begin{theorem}\label{thm:no-interleaving}
    The list specification from \S~\ref{sec:datatypes} does not allow interleaving of concurrent insertions.
    That is, if one user inserts a character sequence $\langle x_1, x_2, \dots, x_n \rangle$ and another user concurrently inserts a character sequence $\langle y_1, y_2, \dots, y_m \rangle$ at the same position, the merged document contains either the character sequence $\langle x_1, x_2, \dots, x_n, y_1, y_2, \dots, y_m \rangle$ or the character sequence $\langle y_1, y_2, \dots, y_m, x_1, x_2, \dots, x_n \rangle$ at the specified position.
\end{theorem}
\begin{proof}
    We formalise the list specification and Theorem~\ref{thm:no-interleaving} using the Isabelle/HOL proof assistant~\cite{DBLP:conf/tphol/WenzelPN08}.
    \ifarxiv
        The formal proof development is summarised in Appendix~\ref{appendix:no-interleaving}.
    \else
        For space reasons, we elide the formal proof development; it is described in the extended version of this paper.
    \fi
\end{proof}

For an informal argument why interleaving is ruled out, see Figure~\ref{fig:op-permutations}, which shows an editing scenario similar to Figure~\ref{fig:bad-merge}, but with the insertions of ``~Alice'' and ``~Charlie'' shortened to ``Al'' and ``Ch'' respectively.
The example contains four insertion operations (``A'', ``l'', ``C'', and ``h''), which can be ordered in six possible ways.
However, among the six possible operation orderings there are only two possible results: \texttt{ChAl} or \texttt{AlCh}.
Interleavings such as \texttt{CAhl} or \texttt{AChl} never occur.

In fact, the end result depends only on the relative ordering of the operations that insert ``A'' and ``C'', respectively.
All other operations can be reordered without affecting the outcome.
Thus, even if the inserted strings are longer than two characters, their relative ordering only depends on the IDs of their first character.
The remaining characters follow their initial character without interleaving.

Note that there are only six possible orderings of the four operations, and not $4! = 24$, because the Lamport timestamp ordering on identifiers (as given in \S~\ref{sec:system-model}) is a linear extension of the causal order.
In this example we assume that text is typed from left to right (that is, ``A'' is always inserted before ``l'', and ``C'' is inserted before ``h'').
This implies that the ID of the operation inserting ``l'' must be greater than that of the insertion of ``A'', and likewise the ``h'' insertion must be greater than the ``C'' insertion.

\begin{figure}
% ``A'', ``l'', ``C'', ``h''
% ``A'', ``C'', ``l'', ``h''
% ``A'', ``C'', ``h'', ``l''
% ``C'', ``A'', ``l'', ``h''
% ``C'', ``A'', ``h'', ``l''
% ``C'', ``h'', ``A'', ``l''
\setlength{\tabcolsep}{1pt}
\begin{tabular}{ll|ll|ll}
$\mathit{id}_1, \mathsf{InsertAfter}(\mathit{id}_0), \text{``A''}$ & $\rightarrow$ \texttt{A} &
$\mathit{id}_1, \mathsf{InsertAfter}(\mathit{id}_0), \text{``A''}$ & $\rightarrow$ \texttt{A} &
$\mathit{id}_1, \mathsf{InsertAfter}(\mathit{id}_0), \text{``A''}$ & $\rightarrow$ \texttt{A} \\
%%%%%%%%%%
$\mathit{id}_2, \mathsf{InsertAfter}(\mathit{id}_1), \text{``l''}$ & $\rightarrow$ \texttt{Al} &
$\mathit{id}_2, \mathsf{InsertAfter}(\mathit{id}_0), \text{``C''}$ & $\rightarrow$ \texttt{CA} &
$\mathit{id}_2, \mathsf{InsertAfter}(\mathit{id}_0), \text{``C''}$ & $\rightarrow$ \texttt{CA} \\
%%%%%%%%%%
$\mathit{id}_3, \mathsf{InsertAfter}(\mathit{id}_0), \text{``C''}$ & $\rightarrow$ \texttt{CAl} &
$\mathit{id}_3, \mathsf{InsertAfter}(\mathit{id}_1), \text{``l''}$ & $\rightarrow$ \texttt{CAl} &
$\mathit{id}_3, \mathsf{InsertAfter}(\mathit{id}_2), \text{``h''}$ & $\rightarrow$ \texttt{ChA} \\
%%%%%%%%%%
$\mathit{id}_4, \mathsf{InsertAfter}(\mathit{id}_3), \text{``h''}$ & $\rightarrow$ \texttt{ChAl} &
$\mathit{id}_4, \mathsf{InsertAfter}(\mathit{id}_2), \text{``h''}$ & $\rightarrow$ \texttt{ChAl} &
$\mathit{id}_4, \mathsf{InsertAfter}(\mathit{id}_1), \text{``l''}$ & $\rightarrow$ \texttt{ChAl} \\[6pt] \hline &&&&&\\[-6pt]
%%%%%%%%%%
$\mathit{id}_1, \mathsf{InsertAfter}(\mathit{id}_0), \text{``C''}$ & $\rightarrow$ \texttt{C} &
$\mathit{id}_1, \mathsf{InsertAfter}(\mathit{id}_0), \text{``C''}$ & $\rightarrow$ \texttt{C} &
$\mathit{id}_1, \mathsf{InsertAfter}(\mathit{id}_0), \text{``C''}$ & $\rightarrow$ \texttt{C} \\
%%%%%%%%%%
$\mathit{id}_2, \mathsf{InsertAfter}(\mathit{id}_0), \text{``A''}$ & $\rightarrow$ \texttt{AC} &
$\mathit{id}_2, \mathsf{InsertAfter}(\mathit{id}_0), \text{``A''}$ & $\rightarrow$ \texttt{AC} &
$\mathit{id}_2, \mathsf{InsertAfter}(\mathit{id}_1), \text{``h''}$ & $\rightarrow$ \texttt{Ch} \\
%%%%%%%%%%
$\mathit{id}_3, \mathsf{InsertAfter}(\mathit{id}_2), \text{``l''}$ & $\rightarrow$ \texttt{AlC} &
$\mathit{id}_3, \mathsf{InsertAfter}(\mathit{id}_1), \text{``h''}$ & $\rightarrow$ \texttt{ACh} &
$\mathit{id}_3, \mathsf{InsertAfter}(\mathit{id}_0), \text{``A''}$ & $\rightarrow$ \texttt{ACh} \\
%%%%%%%%%%
$\mathit{id}_4, \mathsf{InsertAfter}(\mathit{id}_1), \text{``h''}$ & $\rightarrow$ \texttt{AlCh} &
$\mathit{id}_4, \mathsf{InsertAfter}(\mathit{id}_2), \text{``l''}$ & $\rightarrow$ \texttt{AlCh} &
$\mathit{id}_4, \mathsf{InsertAfter}(\mathit{id}_3), \text{``l''}$ & $\rightarrow$ \texttt{AlCh} \\
\end{tabular}
\caption{All possible operation orderings when the strings ``Al'' (for ``Alice'') and ``Ch'' (for ``Charlie'') are concurrently inserted at the same position.
The operation IDs are arbitrary; we only require that $id_0 < id_1 < id_2 < id_3 < id_4$.}\label{fig:op-permutations}
\end{figure}

\begin{theorem}
    The OpSet list specification from \S~\ref{sec:datatypes} is strictly stronger than the $\mathcal{A}_\textsf{strong}$ specification of Attiya et al \cite{Attiya:2016kh}.
    That is, any algorithm that satisfies the list specification given in \S~\ref{sec:datatypes} also satisfies $\mathcal{A}_\textsf{strong}$, but the converse is not true.
\end{theorem}
\begin{proof}
    We formalise the $\mathcal{A}_\textsf{strong}$ specification with Isabelle/HOL, and produce a mechanically verified proof that every possible execution of the list specification from \S~\ref{sec:datatypes} satisfies all conditions of $\mathcal{A}_\textsf{strong}$.
    \ifarxiv
        The formal proof development is summarised in Appendix~\ref{appendix:attiya-spec}.
    \else
        The formal proof development is described in the extended version of this paper.
    \fi
    The fact that our specification is \emph{strictly} stronger follows from Theorems~\ref{thm:attiya-allows-interleaving} and~\ref{thm:no-interleaving}.
\end{proof}

\begin{theorem}
    The RGA algorithm \cite{Roh:2011dw} satisfies the OpSet list specification introduced in this paper, while Logoot \cite{Weiss:2009ht,Weiss:2010hx} and LSEQ \cite{Nedelec:2016eo,Nedelec:2013ky} do not.
\end{theorem}
\begin{proof}
    We use Isabelle/HOL to prove that RGA satisfies our specification, as described in
    \ifarxiv
        Appendix~\ref{appendix:rga}.
    \else
        the extended version of this paper.
    \fi
    Our Isabelle/HOL implementation of RGA is based on the formalisation that we developed in previous work \cite{Gomes:2017vo,Gomes:2017gy}.
    The fact that Logoot and LSEQ do not satisfy our specification follows directly from Theorems~\ref{thm:attiya-allows-interleaving} and~\ref{thm:no-interleaving}.
\end{proof}
