\section{Discussion: Merging Text Edits}\label{sec:bad-merge}

The datatypes we have specified can support a wide range of applications.
For example, the list datatype can be used to implement a collaborative text editor: by treating the text as a list of individual characters, every edit can be expressed as a sequence of insertion or deletion operations on the list.

The problem of collaborative text editing has been studied extensively, both using the Operational Transformation technique \cite{Ellis:1989ue,Nichols:1995fd,Ressel:1996wx,Sun:1998un,Sun:1998vf,Suleiman:1997gl,Suleiman:1998eu,Vidot:2000ch,Imine:2003ks,Li:2004er,Li:2008hw,Oster:2006tr}, and the CRDT approach \cite{Roh:2011dw,Preguica:2009fz,Oster:2006wj,Weiss:2010hx,Nedelec:2013ky,Nedelec:2016eo}.
Attiya et al.~\cite{Attiya:2016kh} recently presented a formal specification of collaborative text editing.
We will now highlight a particular scenario that, to our knowledge, has not been considered by any previous work on collaborative text editing.

Consider the execution illustrated in Figure~\ref{fig:bad-merge}.
In this example, two users are concurrently editing a text document that initially reads ``Hello!''.
The user on the left changes it to read ``Hello Alice!'', while concurrently the user on the right changes the document to read ``Hello Charlie!''.
When the concurrent edits are merged, the algorithm randomly interleaves the two insertions of ``~Alice'' and ``~Charlie'' character by character, resulting in an unreadable jumble of characters.

\begin{figure}
\centering
\begin{tikzpicture}
  \tikzstyle{box}=[rectangle,draw,inner xsep=6pt,text height=9pt,text depth=2pt]
  \tikzstyle{every path}=[draw,-{Stealth[length=3.5mm]}]
  \node [box] (start) at (3,4) {\texttt{Hello!}};
  \node [box] (left)  at (0,2) {\texttt{Hello Alice!}};
  \node [box] (right) at (6,2) {\texttt{Hello Charlie!}};
  \node [box] (merge) at (3,0) {\texttt{Hello Al Ciharcliee!}};
  \draw (start) to node [left,inner xsep=10pt]  {Insert ``~Alice'' between ``o'' and ``!''} (left);
  \draw (start) to node [right,inner xsep=10pt] {Insert ``~Charlie'' between ``o'' and ``!''} (right);
  \draw (left)  -- (merge);
  \draw (right) -- (merge);
  \node [text width=3cm,text badly centered] at (3,1) {Merge concurrent edits};
\end{tikzpicture}
\caption{Two concurrent insertions at the same position are interleaved.}\label{fig:bad-merge}
\end{figure}

The problem is even worse if the concurrent insertions are not just a single word, but an entire paragraph or section.
In these cases, interleaving the users' insertions would most likely result in an entirely incomprehensible text that would have to be deleted and rewritten.
Even though the merge in Figure~\ref{fig:bad-merge} is so obviously undesirable, there is to our knowledge no formal specification of collaborative text editing that rules out such an interleaving of insertions.

\begin{proposition}\label{prop:attiya-allows-interleaving}
    The specification of collaborative text editing by Attiya et al. \cite{Attiya:2016kh} allows the outcome in Figure~\ref{fig:bad-merge}.
    Moreover, the text editing CRDT algorithms Treedoc \cite{Preguica:2009fz}, WOOT \cite{Oster:2006wj}, Logoot \cite{Weiss:2010hx}, and LSEQ \cite{Nedelec:2013ky,Nedelec:2016eo} also allow the outcome in Figure~\ref{fig:bad-merge}.
\end{proposition}
\begin{proof}
    Follows directly from the respective definitions, which are all based on the idea of assigning each character a position in a totally ordered identifier space, such that the order of identifiers corresponds to the order of characters in the document.
    When a new character is inserted, it is assigned an identifier that lies between the identifiers of its predecessor and successor.
    However, when concurrent insertions with the same predecessor and successor are performed, those insertions are ordered arbitrarily.
    Repeated insertions within the same predecessor-successor interval may thus be interleaved arbitrarily.
\end{proof}

Rather than interleaving characters, a better approach to merging is to keep all insertions by a particular user together as a continuous sequence.
With this constraint, there are two acceptable merged results in the example of Figure~\ref{fig:bad-merge}: either ``Hello Alice Charlie!'' or ``Hello Charlie Alice!''.
The choice between these two outcomes is arbitrary, as there is no \emph{a priori} requirement for one user's insertions to come before the other's.

\begin{proposition}\label{prop:no-interleaving}
    The list specification from Section~\ref{sec:datatypes} does not allow the outcome in Figure~\ref{fig:bad-merge}.
\end{proposition}
\begin{proof}
    TODO.
\end{proof}

For an informal argument why interleaving is ruled out, see Figure~\ref{fig:op-permutations}, which shows an editing scenario similar to Figure~\ref{fig:bad-merge}, but with the insertions of ``~Alice'' and ``~Charlie'' shortened to ``Al'' and ``Ch'' respectively.
The example contains four insertion operations (``A'', ``l'', ``C'', and ``h''), which can be ordered in six possible ways.
However, among the six possible operation orderings there are only two possible results: \texttt{ChAl} or \texttt{AlCh}.
Interleavings such as \texttt{CAhl} or \texttt{AChl} never occur.

In fact, the end result depends only on the relative ordering of the operations that insert ``A'' and ``C'', respectively.
All other operations can be reordered without affecting the outcome.
Thus, even if the inserted strings are longer than two characters, their relative ordering only depends on the IDs of their first character.
The remaining characters follow their initial character without interleaving.

Note that there are only six possible orderings of the four operations, and not $4! = 24$, because (as discussed in Section~\ref{sec:system-model}) we require that the ordering on identifiers is a linear extension of the causal order.
In this example we assume that text is typed from left to right (that is, ``A'' is always inserted before ``l'', and ``C'' is inserted before ``h'').
This implies that the ID of the operation inserting ``l'' must be greater than that of the insertion of ``A'', and likewise with ``h'' and ``C''.

\begin{figure}
% ``A'', ``l'', ``C'', ``h''
% ``A'', ``C'', ``l'', ``h''
% ``A'', ``C'', ``h'', ``l''
% ``C'', ``A'', ``l'', ``h''
% ``C'', ``A'', ``h'', ``l''
% ``C'', ``h'', ``A'', ``l''
\setlength{\tabcolsep}{3pt}
\begin{tabular}{ll|ll|ll}
$\mathit{id}_1, \mathsf{InsertAfter}(\mathit{id}_0), \text{``A''}$ & $\rightarrow$ \texttt{A} &
$\mathit{id}_1, \mathsf{InsertAfter}(\mathit{id}_0), \text{``A''}$ & $\rightarrow$ \texttt{A} &
$\mathit{id}_1, \mathsf{InsertAfter}(\mathit{id}_0), \text{``A''}$ & $\rightarrow$ \texttt{A} \\
%%%%%%%%%%
$\mathit{id}_2, \mathsf{InsertAfter}(\mathit{id}_1), \text{``l''}$ & $\rightarrow$ \texttt{Al} &
$\mathit{id}_2, \mathsf{InsertAfter}(\mathit{id}_0), \text{``C''}$ & $\rightarrow$ \texttt{CA} &
$\mathit{id}_2, \mathsf{InsertAfter}(\mathit{id}_0), \text{``C''}$ & $\rightarrow$ \texttt{CA} \\
%%%%%%%%%%
$\mathit{id}_3, \mathsf{InsertAfter}(\mathit{id}_0), \text{``C''}$ & $\rightarrow$ \texttt{CAl} &
$\mathit{id}_3, \mathsf{InsertAfter}(\mathit{id}_1), \text{``l''}$ & $\rightarrow$ \texttt{CAl} &
$\mathit{id}_3, \mathsf{InsertAfter}(\mathit{id}_2), \text{``h''}$ & $\rightarrow$ \texttt{ChA} \\
%%%%%%%%%%
$\mathit{id}_4, \mathsf{InsertAfter}(\mathit{id}_3), \text{``h''}$ & $\rightarrow$ \texttt{ChAl} &
$\mathit{id}_4, \mathsf{InsertAfter}(\mathit{id}_2), \text{``h''}$ & $\rightarrow$ \texttt{ChAl} &
$\mathit{id}_4, \mathsf{InsertAfter}(\mathit{id}_1), \text{``l''}$ & $\rightarrow$ \texttt{ChAl} \\[6pt] \hline &&&&&\\[-6pt]
%%%%%%%%%%
$\mathit{id}_1, \mathsf{InsertAfter}(\mathit{id}_0), \text{``C''}$ & $\rightarrow$ \texttt{C} &
$\mathit{id}_1, \mathsf{InsertAfter}(\mathit{id}_0), \text{``C''}$ & $\rightarrow$ \texttt{C} &
$\mathit{id}_1, \mathsf{InsertAfter}(\mathit{id}_0), \text{``C''}$ & $\rightarrow$ \texttt{C} \\
%%%%%%%%%%
$\mathit{id}_2, \mathsf{InsertAfter}(\mathit{id}_0), \text{``A''}$ & $\rightarrow$ \texttt{AC} &
$\mathit{id}_2, \mathsf{InsertAfter}(\mathit{id}_0), \text{``A''}$ & $\rightarrow$ \texttt{AC} &
$\mathit{id}_2, \mathsf{InsertAfter}(\mathit{id}_1), \text{``h''}$ & $\rightarrow$ \texttt{Ch} \\
%%%%%%%%%%
$\mathit{id}_3, \mathsf{InsertAfter}(\mathit{id}_2), \text{``l''}$ & $\rightarrow$ \texttt{AlC} &
$\mathit{id}_3, \mathsf{InsertAfter}(\mathit{id}_1), \text{``h''}$ & $\rightarrow$ \texttt{ACh} &
$\mathit{id}_3, \mathsf{InsertAfter}(\mathit{id}_0), \text{``A''}$ & $\rightarrow$ \texttt{ACh} \\
%%%%%%%%%%
$\mathit{id}_4, \mathsf{InsertAfter}(\mathit{id}_1), \text{``h''}$ & $\rightarrow$ \texttt{AlCh} &
$\mathit{id}_4, \mathsf{InsertAfter}(\mathit{id}_2), \text{``l''}$ & $\rightarrow$ \texttt{AlCh} &
$\mathit{id}_4, \mathsf{InsertAfter}(\mathit{id}_3), \text{``l''}$ & $\rightarrow$ \texttt{AlCh} \\
\end{tabular}
\caption{All possible operation orderings when ``Al'' (for ``Alice'') and ``Ch'' (for ``Charlie'') are concurrently inserted at the same position.
The IDs are arbitrary; we only require $id_0 < id_1 < id_2 < id_3 < id_4$.}\label{fig:op-permutations}
\end{figure}

\begin{proposition}
    The OpSet list specification introduced in this paper is strictly stronger than the $\mathcal{A}_\textsf{strong}$ specification of Attiya et al \cite{Attiya:2016kh}.
\end{proposition}
\begin{proof}
    We use the Isabelle/HOL proof assistant to show that every datatype implementation that satisfies our list specification also satisfies the specification of Attiya et al.
    The fact that our specification is \emph{strictly} stronger follows from Propositions~\ref{prop:attiya-allows-interleaving} and~\ref{prop:no-interleaving}.

Attiya et al.'s specification is as follows~\cite{Attiya:2016kh}:
\begin{displayquote}
An abstract execution $A = (H, \textsf{vis})$ belongs to the \emph{strong list specification} $\mathcal{A}_\textsf{strong}$ if and only if there is a relation $\textsf{lo} \subseteq \textsf{elems}(A) \times \textsf{elems}(A)$, called the \emph{list order}, such that:
\begin{enumerate}
\item Each event $e = \mathit{do}(\mathit{op}, w) \in H$ returns a sequence of elements $w=a_0 \dots a_{n-1}$, where $a_i \in \textsf{elems}(A)$, such that
\begin{enumerate}
\item $w$ contains exactly the elements visible to $e$ that have been inserted, but not deleted:
\[ \forall a.\; a \in w \quad\Longleftrightarrow\quad (\mathit{do}(\textsf{ins}(a, \_), \_) \le_\textsf{vis} e)
\;\wedge\; \neg(\mathit{do}(\textsf{del}(a), \_) \le_\textsf{vis} e). \]
\item The order of the elements is consistent with the list order:
\[ \forall i, j.\; (i < j) \;\Longrightarrow\; (a_i, a_j) \in \textsf{lo}. \]
\item Elements are inserted at the specified position:
if $\mathit{op} = \textsf{ins}(a, k)$, then $a = a_{\mathrm{min} \{k,\; n-1\}}$.
\end{enumerate}
\item The list order $\textsf{lo}$ is transitive, irreflexive and total, and thus determines the order of all insert operations in the execution.
\end{enumerate}
\end{displayquote}

The expression $\mathit{do}(\mathit{op}, w)$ denotes an operation that is performed by a user on one of the replicas.
While our specification is based on operations being replicated into other nodes' OpSets, Attiya et al.'s specification instead uses a visibility relation $\le_\textsf{vis}$.
We formalised the above specification in Isabelle/HOL, and prove formally that our specification is strictly stronger.
\end{proof}
