\section{Discussion}

% Bieniusa et al.'s principle of permutation equivalence \cite{Bieniusa:2012gt}

This sequential interpretation of operations, in order of $\mathit{id}$, is reminiscent of the concept of \emph{serializability} in databases: the data structure obtained by interpreting an OpSet is equal to the outcome of applying the operations in their serial order, even if the execution that produced the OpSet was in fact concurrent.
Reducing concurrent executions to a serial order makes them much easier to understand.

%Since we know that the query result is determined entirely by the OpSet, we know that even if the operations arrive at a node in any arbitrary order, the final query result must be the same as if they had been applied sequentially in order of ascending identifier.

%In other words, we have \emph{operation serializability}: 
%It is therefore sufficient for us to define the semantics of each operation under serial execution (in order of ascending operation ID), and we know that this definition will also define its semantics in arbitrary concurrent executions: any state in a concurrent execution corresponds to some OpSet, and the operations in that OpSet are serializable.

\section{Related Work}\label{sec:relwork}


Algorithms to achieve this goal have been the topic of active research for approximately 30 years, under the headings of operational transformation \cite{Ellis:1989ue,Li:2004er,Nichols:1995fd,Oster:2006tr,Ressel:1996wx,Suleiman:1998eu,Sun:1998vf,Sun:1998un} and CRDTs \cite{Nedelec:2013ky,Nedelec:2016eo,Oster:2006wj,Preguica:2009fz,Roh:2011dw,Weiss:2010hx}.

However, throughout this time, the exact consistency properties provided by the algorithms have been somewhat unclear.
For example, Sun et al.~\cite{Sun:1998un} identified three desirable properties that they articulated informally: \emph{convergence}, \emph{causality preservation}, and \emph{intention preservation}.
While the definition of the first two properties is fairly unambiguous, the definition of ``intention preservation'' leaves much more room for interpretation.
Sun et al.\ define it as follows \cite{Sun:1998un}:
\begin{displayquote}
For any operation $O$, the effects of executing $O$ at all sites are the same as the intention of $O$, and the effect of executing $O$ does not change the effects of independent operations.
\end{displayquote}
where the term \emph{intention} is in turn defined as:
\begin{displayquote}
The intention of an operation $O$ is the execution effect which can be achieved by applying $O$ on the document state from which $O$ was generated.
\end{displayquote}
We now show how to define consistency properties for a list in a way that is simple and unambiguous, using the OpSets approach.

