\section{Specifying a Collaborative List}\label{sec:list}

An ordered list datatype (also known as \emph{sequence}) is the foundation of many collaborative applications.
For example, a text document can be expressed as an ordered list of characters, and an XML document is a tree in which the contents of each node is an ordered list of child nodes.

The basic operations on a list are the insertion and the deletion of individual list elements.
When several users concurrently modify a shared list object, an algorithm is required to merge these changes while ensuring that certain consistency properties are satisfied.
Algorithms to achieve this goal have been the topic of active research for approximately 30 years, under the headings of operational transformation \cite{Ellis:1989ue,Li:2004er,Nichols:1995fd,Oster:2006tr,Ressel:1996wx,Suleiman:1998eu,Sun:1998vf,Sun:1998un} and CRDTs \cite{Nedelec:2013ky,Nedelec:2016eo,Oster:2006wj,Preguica:2009fz,Roh:2011dw,Weiss:2010hx}.

However, throughout this time, the exact consistency properties provided by the algorithms have been somewhat unclear.
For example, Sun et al.~\cite{Sun:1998un} identified three desirable properties that they articulated informally: \emph{convergence}, \emph{causality preservation}, and \emph{intention preservation}.
While the definition of the first two properties is fairly unambiguous, the definition of ``intention preservation'' leaves much more room for interpretation.
Sun et al.\ define it as follows \cite{Sun:1998un}:
\begin{displayquote}
For any operation $O$, the effects of executing $O$ at all sites are the same as the intention of $O$, and the effect of executing $O$ does not change the effects of independent operations.
\end{displayquote}
where the term \emph{intention} is in turn defined as:
\begin{displayquote}
The intention of an operation $O$ is the execution effect which can be achieved by applying $O$ on the document state from which $O$ was generated.
\end{displayquote}
We now show how to define consistency properties for a list in a way that is simple and unambiguous, using the OpSets approach.

\subsection{Sequential Single-Element Insertion and Deletion}

we require several consistency guarantees that we informally introduced in Section~\ref{sec:bad-merge}.


