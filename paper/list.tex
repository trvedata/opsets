\section{Specifying a Collaborative List}\label{sec:list}

%An ordered list datatype (also known as \emph{sequence}) is the foundation of many collaborative applications.
%For example, a text document can be expressed as an ordered list of characters, and an XML document is a tree in which the contents of each node is an ordered list of child nodes.

%The basic operations on a list are the insertion and the deletion of individual list elements.
%When several users concurrently modify a shared list object, an algorithm is required to merge these changes while ensuring that certain consistency properties are satisfied.

\noindent
\renewcommand\algorithmicindent{10pt}
\begin{minipage}[t]{0.5\textwidth}
\begin{algorithmic}
    \Function{setMapKey}{$O, \mathit{map}, \mathit{key}, \mathit{val}$}
    \State $(F,L) = \llbracket O \rrbracket$
    \State $(\mathit{id}_1, \mathit{op}_1) = \Call{valueID}{O, \mathit{val}}$
    \State $O' = (\textbf{if } \mathit{op}_1 = \bot \textbf{ then } O \textbf{ else } 
        O \;\cup\; \big\{ (\mathit{id}_1, \mathit{op}_1) \big\})$
    \State $\mathit{id}_2 = \mathrm{newID}(O')$
    \State $\mathit{prev} = \{ \mathit{id} \mid \exists\,\mathit{val}.\; (\mathit{id}, \mathit{map}, \mathit{key}, \mathit{val}) \in F \}$
    \State \Return $O' \;\cup\; \big\{ (\mathit{id}_2,\, \mathsf{Assign}(\mathit{map}, \mathit{key}, \mathit{id}_1, \mathit{prev})) \big\}$
    \EndFunction\Statex

    \Function{removeMapKey}{$O, \mathit{map}, \mathit{key}$}
    \State $(F,L) = \llbracket O \rrbracket$
    \State $\mathit{id}_1 = \mathrm{newID}(O)$
    \State $\mathit{prev} = \{ \mathit{id} \mid \exists\,\mathit{val}.\; (\mathit{id}, \mathit{map}, \mathit{key}, \mathit{val}) \in F \}$
    \State \Return $O \;\cup\; \big\{ (\mathit{id}_1,\, \mathsf{Remove}(\mathit{map}, \mathit{key}, \mathit{prev})) \big\}$
    \EndFunction\Statex

    \Function{valueID}{$O, \mathit{val}$}
    \If{$\mathit{val}$ is a primitive type}
    \State \Return $(\mathrm{newID}(O),\, \mathsf{MakeVal}(\mathit{val}))$
    \ElsIf{$\mathit{val} = \texttt{[]}$}
    \State \Return $(\mathrm{newID}(O),\, \mathsf{MakeList})$
    \ElsIf{$\mathit{val} = \texttt{{\char '173}{\char '175}}$}
    \State \Return $(\mathrm{newID}(O),\, \mathsf{MakeMap})$
    \Else ~// $\mathit{val}$ is an existing object
    \State \Return $(\mathrm{objID}(\mathit{val}),\, \bot)$
    \EndIf
    \EndFunction
\end{algorithmic}
\end{minipage}%
\begin{minipage}[t]{0.5\textwidth}
\begin{algorithmic}
    \Function{setListIndex}{$O, \mathit{list}, \mathit{index}, \mathit{val}$}
    \State $(F,L) = \llbracket O \rrbracket$
    \State $(\mathit{id}_1, \mathit{op}_1) = \Call{valueID}{O, \mathit{val}}$
    \State $O' = (\textbf{if } \mathit{op}_1 = \bot \textbf{ then } O \textbf{ else } 
        O \;\cup\; \big\{ (\mathit{id}_1, \mathit{op}_1) \big\})$
    \State $\mathit{id}_2 = \mathrm{newID}(O')$
    \State $\mathit{key} = \mathrm{idxKey}_{F,L}(\mathit{list}, \mathit{list}, \mathit{index})$
    \State $\mathit{prev} = \{ \mathit{id} \mid \exists\,\mathit{val}.\; (\mathit{id}, \mathit{list}, \mathit{key}, \mathit{val}) \in F \}$
    \State \Return $O' \;\cup\; \big\{ (\mathit{id}_2,\, \mathsf{Assign}(\mathit{list}, \mathit{key}, \mathit{id}_1, \mathit{prev})) \big\}$
    \EndFunction\Statex

    \Function{insertListIndex}{$O, \mathit{list}, \mathit{index}, \mathit{val}$}
    \State $(F,L) = \llbracket O \rrbracket$
    \State $(\mathit{id}_1, \mathit{op}_1) = \Call{valueID}{O, \mathit{val}}$
    \State $O' = (\textbf{if } \mathit{op}_1 = \bot \textbf{ then } O \textbf{ else } 
        O \;\cup\; \big\{ (\mathit{id}_1, \mathit{op}_1) \big\})$
    \State $\mathit{id}_2 = \mathrm{newID}(O')$
    \State $\mathit{ref} = (\textbf{if } \mathit{index}=0 \textbf{ then } \mathit{list}$
    \State $\hphantom{\mathit{ref} = (}\textbf{else } \mathrm{idxKey}_{F,L}(\mathit{list}, \mathit{list}, \mathit{index} - 1))$
    \State $O'' = O' \;\cup\; \big\{ (\mathit{id}_2,\, \mathsf{InsertAfter}(\mathit{ref})) \big\}$
    \State $\mathit{id}_3 = \mathrm{newID}(O'')$
    \State \Return $O'' \;\cup\; \big\{ (\mathit{id}_3,\, \mathsf{Assign}(\mathit{list}, \mathit{id}_2, \mathit{id}_1, \emptyset)) \big\}$
    \EndFunction\Statex

    \Function{removeListIndex}{$O, \mathit{list}, \mathit{index}$}
    \State $(F,L) = \llbracket O \rrbracket$
    \State $\mathit{id}_1 = \mathrm{newID}(O)$
    \State $\mathit{key} = \mathrm{idxKey}_{F,L}(\mathit{list}, \mathit{list}, \mathit{index})$
    \State $\mathit{prev} = \{ \mathit{id} \mid \exists\,\mathit{val}.\; (\mathit{id}, \mathit{list}, \mathit{key}, \mathit{val}) \in F \}$
    \State \Return $O \;\cup\; \big\{ (\mathit{id}_1,\, \mathsf{Remove}(\mathit{list}, \mathit{key}, \mathit{prev})) \big\}$
    \EndFunction
\end{algorithmic}
\end{minipage}

\[ \mathrm{idxKey}_{F,L}(\mathit{obj}, \mathit{key}, i) \;=\; \left\{
   \arraycolsep=2pt \def\arraystretch{1.3}
   \begin{array}{llllll}
       \mathrm{idxKey}_{F,L}(\mathit{obj}, n, i-1) &
       \quad\text{if }\; i > 0 & \wedge & (\mathit{key}, n) \in L & \wedge &
       \exists\,\mathit{id}, \mathit{val}.\; (\mathit{id}, \mathit{obj}, \mathit{key}, \mathit{val}) \in F \\
       \mathrm{idxKey}_{F,L}(\mathit{obj}, n, i) &
       \quad\text{if }\; && (\mathit{key}, n) \in L & \wedge &
       \nexists\,\mathit{id}, \mathit{val}.\; (\mathit{id}, \mathit{obj}, \mathit{key}, \mathit{val}) \in F \\
       \mathit{key} &
       \quad\text{if }\; i = 0 & \wedge &&&
       \exists\,\mathit{id}, \mathit{val}.\; (\mathit{id}, \mathit{obj}, \mathit{key}, \mathit{val}) \in F \\
   \end{array} \right. \]

\begin{align*}
    \llbracket O \rrbracket &= \llbracket O \rrbracket_{\emptyset, \emptyset} \\
    \llbracket \emptyset \rrbracket_{F,L} &= (F,L) \\
    \llbracket O \rrbracket_{F,L} &= \llbracket O - \{\mathit{op}\} \rrbracket_{F',L'} &&
    \text{where } O \neq \emptyset \;\wedge\; \mathit{op} = \mathrm{minOp}(O) \;\wedge\;
    (F',L') = \llbracket \mathit{op} \rrbracket_{F,L} \\
    \mathrm{minOp}(O) &= (\mathit{id}, \mathit{op}) &&
    \text{where } (\mathit{id}, \mathit{op}) \in O \;\wedge\;
    \nexists\,(\mathit{id}', \mathit{op}') \in O.\; \mathit{id}' < \mathit{id}
\end{align*}

\begin{align*}
    \big\llbracket (\mathit{id},\, \mathsf{Assign}(\mathit{obj}, \mathit{key}, \mathit{val}, \mathit{prev})) \big\rrbracket_{F,L} \;=\; \Big( &
    \big\{ (\mathit{id}', \mathit{obj}', \mathit{key}', \mathit{val}') \in F \mid
    \mathit{id}' \notin \mathit{prev} \big\} \;\cup\;
    \big\{ (\mathit{id}, \mathit{obj}, \mathit{key}, \mathit{val}) \big\},\; L \Big) \\[5pt]
    %%%%%%%%%%
    \big\llbracket (\mathit{id},\, \mathsf{Remove}(\mathit{obj}, \mathit{key}, \mathit{prev})) \big\rrbracket_{F,L} \;=\; \Big( &
    \big\{ (\mathit{id}', \mathit{obj}', \mathit{key}', \mathit{val}') \in F \mid
    \mathit{id}' \notin \mathit{prev} \big\},\; L \Big) \\[5pt]
    %%%%%%%%%%
    \big\llbracket (\mathit{id},\, \mathsf{InsertAfter}(\mathit{ref})) \big\rrbracket_{F,L} \;=\; \Big( & F,\;
    \big\{ (p,n) \in L \mid p \neq \mathit{ref} \big\} \;\cup\;
    \big\{ (\mathit{ref}, \mathit{id}) \big\} \;\cup\;
    \big\{ (\mathit{id}, n) \mid (\mathit{ref}, n) \in L \big\} \Big)
\end{align*}

\subsection{Specification versus Implementation}

The definition is almost trivially simple.
And yet, as we shall argue in the following sections, it is the most precise specification that has been given for a collaboratively editable list, and it subsumes prior specifications given in the literature.

Note that $\isa{interp{\isacharunderscore}list}$ is defined only for a list of operations that is sorted in order of ascending oid.
For any given set of operations, this sorted list is uniquely defined, and thus the interpretation is a deterministic function of the OpSet.
As discussed in Section~\ref{sec:approach}, this fact implies that the specification is convergent: as two nodes converge towards the same set of operations, their interpretation of the OpSet also converges.

However, in a system with several concurrently acting nodes, it is likely that the order in which nodes receive operations is not consistent with the oid order.
When a node receives an operation with a lower oid than an operation already applied, the $\isa{interp}$ function cannot be directly used to apply the operation to the interpreted node state, because $\isa{interp}$ is not commutative.
Instead, $\isa{interp{\isacharunderscore}list}$ must replay all operations in ascending order, incorporating the new operation at the appropriate point in the sorted list.
(Alternatively, a node could keep checkpoints of the interpreted state at selected older oids, jump back to the most recent checkpoint that precedes the oid of the incoming operation, and replay operations from that point onward.)

Viewed in this way, the specification is not an efficient implementation of a replicated datatype.
However, it is a simple, unambiguous, and executable specification of the operation semantics.
We can now produce more efficient implementations that allow operations to be processed out of order, and prove that their output is consistent with the specification.

\subsection{Preventing Interleaving}

An important property of the specification is that the interleaving anomaly of Figure~\ref{fig:bad-merge} is not allowed.
It is not immediately obvious why this is the case, but we prove this fact in Isabelle/HOL.

For an informal, intuitive argument why interleaving is ruled out, see Figure~\ref{fig:op-permutations}, which shows an editing scenario similar to Figure~\ref{fig:bad-merge}, but with the insertions of ``~Alice'' and ``~Charlie'' shortened to ``Al'' and ``Ch'' respectively.
The example contains four insertion operations (``A'', ``l'', ``C'', and ``h''), which can be ordered in six possible ways.
However, among the six possible operation orderings there are only two possible results: \texttt{ChAl} or \texttt{AlCh}.
Interleavings such as \texttt{CAhl} or \texttt{AChl} never occur.

In fact, the end result depends only on the relative ordering of the operations that insert ``A'' and ``C'', respectively.
All other operations can be reordered without affecting the outcome.
Thus, even if the inserted strings are longer than two characters, their relative ordering only depends on the oids of the first character of the strings.
The rest of the strings follow their initial character without interleaving.

Note that there are only six possible orderings of the four operations, and not $24 = 4!$, because (as discussed in Section~\ref{sec:op-serial}) we require that the ordering on identifiers is a linear extension of the causal order.
In this example we assume that text is typed from left to right (that is, ``A'' is always inserted before ``l'', and ``C'' is inserted before ``h'').
This implies that the oid of the operation inserting ``l'' must be greater than that of the insertion of ``A'', and likewise with ``h'' and ``C''.

\begin{figure}
% ``A'', ``l'', ``C'', ``h''
% ``A'', ``C'', ``l'', ``h''
% ``A'', ``C'', ``h'', ``l''
% ``C'', ``A'', ``l'', ``h''
% ``C'', ``A'', ``h'', ``l''
% ``C'', ``h'', ``A'', ``l''
\setlength{\tabcolsep}{3pt}
\begin{tabular}{ll|ll|ll}
$(\mathit{id}_1, \isa{InsertAfter } \mathit{id}_0, \text{``A''})$ & $\rightarrow$ \texttt{A} &
$(\mathit{id}_1, \isa{InsertAfter } \mathit{id}_0, \text{``A''})$ & $\rightarrow$ \texttt{A} &
$(\mathit{id}_1, \isa{InsertAfter } \mathit{id}_0, \text{``A''})$ & $\rightarrow$ \texttt{A} \\
%%%%%%%%%%
$(\mathit{id}_2, \isa{InsertAfter } \mathit{id}_1, \text{``l''})$ & $\rightarrow$ \texttt{Al} &
$(\mathit{id}_2, \isa{InsertAfter } \mathit{id}_0, \text{``C''})$ & $\rightarrow$ \texttt{CA} &
$(\mathit{id}_2, \isa{InsertAfter } \mathit{id}_0, \text{``C''})$ & $\rightarrow$ \texttt{CA} \\
%%%%%%%%%%
$(\mathit{id}_3, \isa{InsertAfter } \mathit{id}_0, \text{``C''})$ & $\rightarrow$ \texttt{CAl} &
$(\mathit{id}_3, \isa{InsertAfter } \mathit{id}_1, \text{``l''})$ & $\rightarrow$ \texttt{CAl} &
$(\mathit{id}_3, \isa{InsertAfter } \mathit{id}_2, \text{``h''})$ & $\rightarrow$ \texttt{ChA} \\
%%%%%%%%%%
$(\mathit{id}_4, \isa{InsertAfter } \mathit{id}_3, \text{``h''})$ & $\rightarrow$ \texttt{ChAl} &
$(\mathit{id}_4, \isa{InsertAfter } \mathit{id}_2, \text{``h''})$ & $\rightarrow$ \texttt{ChAl} &
$(\mathit{id}_4, \isa{InsertAfter } \mathit{id}_1, \text{``l''})$ & $\rightarrow$ \texttt{ChAl} \\[6pt] \hline &&&&&\\[-6pt]
%%%%%%%%%%
$(\mathit{id}_1, \isa{InsertAfter } \mathit{id}_0, \text{``C''})$ & $\rightarrow$ \texttt{C} &
$(\mathit{id}_1, \isa{InsertAfter } \mathit{id}_0, \text{``C''})$ & $\rightarrow$ \texttt{C} &
$(\mathit{id}_1, \isa{InsertAfter } \mathit{id}_0, \text{``C''})$ & $\rightarrow$ \texttt{C} \\
%%%%%%%%%%
$(\mathit{id}_2, \isa{InsertAfter } \mathit{id}_0, \text{``A''})$ & $\rightarrow$ \texttt{AC} &
$(\mathit{id}_2, \isa{InsertAfter } \mathit{id}_0, \text{``A''})$ & $\rightarrow$ \texttt{AC} &
$(\mathit{id}_2, \isa{InsertAfter } \mathit{id}_1, \text{``h''})$ & $\rightarrow$ \texttt{Ch} \\
%%%%%%%%%%
$(\mathit{id}_3, \isa{InsertAfter } \mathit{id}_2, \text{``l''})$ & $\rightarrow$ \texttt{AlC} &
$(\mathit{id}_3, \isa{InsertAfter } \mathit{id}_1, \text{``h''})$ & $\rightarrow$ \texttt{ACh} &
$(\mathit{id}_3, \isa{InsertAfter } \mathit{id}_0, \text{``A''})$ & $\rightarrow$ \texttt{ACh} \\
%%%%%%%%%%
$(\mathit{id}_4, \isa{InsertAfter } \mathit{id}_1, \text{``h''})$ & $\rightarrow$ \texttt{AlCh} &
$(\mathit{id}_4, \isa{InsertAfter } \mathit{id}_2, \text{``l''})$ & $\rightarrow$ \texttt{AlCh} &
$(\mathit{id}_4, \isa{InsertAfter } \mathit{id}_3, \text{``l''})$ & $\rightarrow$ \texttt{AlCh} \\
\end{tabular}
\caption{All possible operation orderings when ``Al'' (for ``Alice'') and ``Ch'' (for ``Charlie'') are concurrently inserted.
The IDs are arbitrary; we only require $id_0 < id_1 < id_2 < id_3 < id_4$.}\label{fig:op-permutations}
\end{figure}

\subsection{Comparison with Earlier Specifications}

We are aware of only one other formal specification of a collaboratively editable list datatype, which is due to Attiya et al.~\cite{Attiya:2016kh}
This specification is as follows:
\begin{displayquote}
An abstract execution $A = (H, \textsf{vis})$ belongs to the \emph{strong list specification} $\mathcal{A}_\textsf{strong}$ if and only if there is a relation $\textsf{lo} \subseteq \textsf{elems}(A) \times \textsf{elems}(A)$, called the \emph{list order}, such that:
\begin{enumerate}
\item Each event $e = \mathit{do}(\mathit{op}, w) \in H$ returns a sequence of elements $w=a_0 \dots a_{n-1}$, where $a_i \in \textsf{elems}(A)$, such that
\begin{enumerate}
\item $w$ contains exactly the elements visible to $e$ that have been inserted, but not deleted:
\[ \forall a.\; a \in w \quad\Longleftrightarrow\quad (\mathit{do}(\textsf{ins}(a, \_), \_) \le_\textsf{vis} e)
\;\wedge\; \neg(\mathit{do}(\textsf{del}(a), \_) \le_\textsf{vis} e). \]
\item The order of the elements is consistent with the list order:
\[ \forall i, j.\; (i < j) \;\Longrightarrow\; (a_i, a_j) \in \textsf{lo}. \]
\item Elements are inserted at the specified position:
if $\mathit{op} = \textsf{ins}(a, k)$, then $a = a_{\mathrm{min} \{k,\; n-1\}}$.
\end{enumerate}
\item The list order $\textsf{lo}$ is transitive, irreflexive and total, and thus determines the order of all insert operations in the execution.
\end{enumerate}
\end{displayquote}

The expression $\mathit{do}(\mathit{op}, w)$ denotes an operation that is performed by a user on one of the replicas.
While our specification is based on operations being replicated into other nodes' OpSets, Attiya et al.'s specification instead uses a visibility relation $\le_\textsf{vis}$.

We formalised the above specification in Isabelle/HOL, and prove formally that our specification is strictly stronger.
That is, every algorithm that satisfies our specification is guaranteed to also satisfy Attiya et al.'s specification.
However, there are also algorithms that satisfy Attiya et al.'s specification but not ours.
This is because Attiya et al.\ allow the interleaving behavior shown in Figure~\ref{fig:bad-merge}, whereas our specification does not allow it.

TODO more detail\dots
