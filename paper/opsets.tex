\documentclass[twocolumn,10pt]{article}
\usepackage[a4paper,margin=2cm]{geometry}
\usepackage[utf8]{inputenc}
\usepackage{mathptmx} % times roman, including math
\usepackage[hyphens]{url}
\usepackage{doi}
\usepackage{hyperref}
\usepackage[numbers,sort]{natbib}
\usepackage{amsmath}
\hyphenation{da-ta-cen-ter da-ta-cen-ters}
\frenchspacing

\begin{document}
\sloppy
\title{OpSets: Concurrent Datatypes with Sequential Specifications}
\author{}
\date{}
\maketitle

\subsection*{Abstract}

\section{Introduction}

A common requirement across a great variety of systems is that several participants may concurrently access and manipulate some shared data structure.
However, unconstrained concurrent manipulation may introduce inconsistencies that applications cannot tolerate.
Thus, various approaches for providing \emph{consistency guarantees} have been developed, ensuring that the data structure continues to obey certain invariants or semantics under concurrent access. For example:

\begin{description}
\item[Transaction isolation] in databases restricts the degree to which concurrent transactions can affect each other while accessing the same database.
The strongest isolation level, \emph{serializability}, ensures that transactions behave as if there were no concurrency at all, i.e.\ as if transactions were executed serially, one at a time.

\item[Conflict-free Replicated Data Types (CRDTs)] allow each participant to modify a local copy (\emph{replica}) of a shared data structure without waiting for communication with other replicas.
This has the consequence that the state of replicas may temporarily diverge, but the definition of CRDTs ensures that all replicas eventually converge towards a consistent state.

\item[Operational Transformation (OT)] algorithms are designed for several users who are collaboratively editing a shared document, as implemented for example in Google Docs.
As with CRDTs, OT allows each user's changes to be applied immediately to their local copy, while ensuring that other users' concurrent edits can be integrated in a consistent way.
\end{description}

We discuss these techniques in more detail in Section~\ref{sec:relwork}.
Despite decades of research in the above topics, consistency mechanisms for concurrent data access remain poorly understood, and tools for reasoning about them (both formally and informally) are subtle and error-prone.

In this work we introduce \emph{Operation Sets} (or \emph{OpSets} for short), a novel approach for describing and reasoning about algorithms for concurrent data access and manipulation.
The OpSets approach trivially ensures that replicas converge, and it simplifies the tasks of extending algorithms and verifying more complex correctness properties.
It builds upon existing work on CRDTs, and extends it by incorporating ideas from database query languages, transaction isolation, and operational transformation.

Our contributions in this paper are as follows:

\begin{itemize}
\item We demonstrate how to express a variety of abstract datatypes (maps, lists, trees, and registers) as declarative definitions on top of a simple abstraction, the OpSet.
By using the Datalog query language for our definitions, we strike an effective compromise between two competing requirements: being abstract enough to make formal verification feasible, and being concrete enough to enable efficient practical implementations.

\item We define correctness properties of the above abstract datatypes using a generalised notion of serializability, and we use the Isabelle/HOL proof assistant to formally verify that our implementation satisfies those correctness properties.
Furthermore, we demonstrate that our correctness properties capture intuitive requirements by giving counter-examples for several existing algorithms in the literature, showing that they do not satisfy the required semantics.

\item To demonstrate the ease of extending OpSet definitions with new operations, we provide an implementation of a \emph{move} operation for lists and trees~-- an operation that is not supported by prior CRDT algorithms~-- and we verify its correctness against a specification.

\item We show how to perform garbage collection on OpSets to prevent unbounded growth on a dataset. Unlike several prior proposals for garbage collection in CRDTs, our approach does not require any global coordination or consensus between nodes.

\item We discuss the distinction between three families of CRDTs in the literature (operation-based, state-based, and delta-based), and argue that our OpSet approach makes this distinction largely redundant: we require only a basic set CRDT as foundation, and show that other abstract datatypes can efficiently be implemented on top of that set, without making further assumptions about the properties of the underlying network.
\end{itemize}

\section{The OpSets Approach}\label{sec:approach}

At a high level, consistency models for distributed data systems can be classified into two categories:
\begin{description}
\item[Strong consistency models] such as serialisability or linearisability \cite{Herlihy:1990jq} attempt to make a system behave like a single sequentially executing node, even when it is in fact replicated and concurrent.
An unavoidable downside of these models is that they require waiting for synchronous network communication before any operation or transaction is allowed to complete \cite{Davidson:1985hv,Gilbert:2002il}.
Thus, a node cannot make progress while it is offline or partitioned from other nodes.
In a formal sense, enforcing a strong consistency model is equivalent to consensus \cite{Chandra:1996cp,Herlihy:1991gk}, which implies that the algorithm cannot be guaranteed to terminate in an asynchronous system \cite{Fischer:1985tt}.

\item[Weak consistency models] are employed by systems that prioritise availability over strong consistency; for example, systems that require nodes to be able to make progress while offline or while partitioned from other nodes.
In such systems, each node typically reads and manipulates a local copy of the shared state, and propagates any changes to that state asynchronously to other nodes.
Examples of weak consistency models in this category are \emph{causal consistency} \cite{Attiya:2015dm,Mahajan:2011wz,Lloyd:2011hz} and \emph{eventual consistency} \cite{Bailis:2013jc,Burckhardt:2014hy,Terry:1994fp,Vogels:2009ca}.
\end{description}

In this work we focus on the latter category of weak consistency models.
However, as we shall demonstrate shortly, the OpSets approach allows us to specify our consistency model more precisely than existing techiques permit.

\subsection{Eventual Consistency is Insufficient}\label{sec:eventual-consistency}

Eventual consistency is usually informally defined as follows: \emph{if no new updates are made to the shared state, all nodes will eventually have the same data}.
This is a very weak definition for several reasons:
\begin{itemize}
\item The premise, \emph{if no new updates are made}, may never be true (if the shared state is continually modified because the system is never quiescent).
In that case, eventual consistency becomes a vacuous statement.

\item Many trivial algorithms satisfy the requirement of converging towards the same state.
For example, a system could simply discard writes, and thus converge to a state in which operations have been ignored.
Although such a system would not be useful in practice, the definition of eventual consistency does not capture the requirement that writes should be persistent.

\item In a system that allows nodes to make progress while they are partitioned from other nodes, it is inevitable that the local state of individual nodes might diverge due to concurrent modifications.
Such divergent states must then be merged at a later time, when communication between the nodes is restored.
Even if we require that this merge operation does not discard data that has been written, eventual consistency does not specify which states should be considered valid results of the merge operation.
\end{itemize}

\subsection{Introducing OpSets}\label{sec:opsets-intro}

The OpSets approach is a simple abstraction that allows us to be more precise about the consistency properties of a replicated data system.
We assume that the system consists of a set of \emph{nodes} connected by a network.
These nodes concurrently access some \emph{shared data structure}, which may be a relational database (consisting of rows in tables), a text document (an ordered list of characters), a vector-graphics document (a tree of records describing graphical objects), or any other kind of data structure.

We assume that each node has a local copy of the shared data structure, which it can read and modify without locking or any other coordination with other nodes.
Whenever a node makes a modification to that structure, it records the change as an \emph{operation}.
For example, an operation may describe a particular insertion at a particular position in a document.
We assume that each operation has a unique identifier that is different from any other operation generated anywhere in the system.
For example, the identifier may consist of a unique node identifier and a sequence number.

Each node locally maintains a set of operations, the \emph{OpSet}.
Whenever a node makes a change, it adds the corresponding operation to its OpSet, and also broadcasts the operation to other nodes.
Whenever a node receives an operation from another node, that operation is also added to the recipient's local OpSet.
Any operations that are lost in the network are retransmitted as necessary.
Operations remain immutable throughout this process.

Thus, the OpSet is a monotonically growing set of operations, and every operation is eventually contained in the OpSet of every node from which it is not permanently partitioned.
(To deal with unbounded growth, we discuss garbage collection later in this paper.)
Any two communicating nodes can merge their OpSets using the standard set union, which is commutative, associative, and idempotent, ensuring that communicating nodes converge towards the same OpSet contents.

\subsection{Data Structures as Queries}\label{sec:queries}

Existing algorithms for maintaining replicated state, such as CRDTs, describe how a node's local state may be manipulated as a result of operations.
We now depart from this convention and present an alternative formulation of replicated data structures.

In the OpSets approach, we require that the shared data structure is never manipulated directly.
Instead, we assume the existence of a pure function that takes an OpSet as input, and returns the current state of the shared data structure according to the OpSet.
The transformation from OpSet to data structure is deterministic and depends only on the contents of the OpSet -- regardless of whether that data structure is a relational database, a text document, a tree, a graph, or anything else.
All nodes in the system employ the same transformation function.
Consequently, whenever any two nodes have the same OpSet, their view of the shared data structure must also be equal.

One can regard the OpSet of being a \emph{database of facts}, containing all of the changes ever made to the shared data.
The function that deterministically transforms this database into some derived data structure is a \emph{query}.
The resulting data structure is, in database terminology, a \emph{materialized view} onto the underlying set of operations.

We show in the following sections how many practical data structures can be expressed as queries.
For now we note that this construction trivially ensures eventual consistency: as two nodes converge towards the same OpSet contents, any data structure that is deterministically derived from the OpSet must also converge.
By using a deterministic query language to derive the data structure from the OpSet, we can~-- by construction~-- rule out any violations of this convergence property.

Moreover, we can take advantage of a rich body of existing research on query languages and materialized view maintenance.
When a new operation is added to the OpSet, a query execution engine can determine the change to the derived data structure that results from the addition of the new operation.
Determining this change to the query result is known as \emph{incremental view maintenance}, and we build upon extensive prior research in this area.

\subsection{Operation Serializability}\label{sec:op-serial}

Like in the definition of eventual consistency in Section~\ref{sec:eventual-consistency}, deriving a data structure from an OpSet ensures convergence, but it does not define how an operation should take effect.
To refine our correctness properties, we must define the expected semantics of operations.

Normally, reasoning about the semantics of concurrently modified data structures is difficult and error-prone.
However, the OpSets model enables us to reason about operation semantics in a sequential way, and directly extend the sequential semantics to arbitrary concurrent executions.

We previously required that every operation has a unique identifier.
We now assume that we also have a total ordering on operation identifiers, and that this total order is a linear extension of the partial order that captures their causal dependencies (the \emph{happens-before} relation).
That is, whenever a node generates a new operation, the identifier of the new operation must be greater than that of any existing operation in the OpSet of the node generating the new operation.
This requirement can easily be met by using Lamport timestamps \cite{Lamport:1978jq} as identifiers.

Now observe that for any OpSet there exists a unique sequence of operations, containing all operations of the OpSet in ascending order of their identifier.
We can specify the semantics of each operation~-- that is, the effect of the operation on the query result~-- when applied in this sequential order.
Since we know that the query result is determined entirely by the OpSet, we know that even if the operations arrive at a node in any arbitrary order, the final query result must be the same as if they had been applied sequentially in order of ascending identifier.

In other words, we have \emph{operation serializability}: the data structure derived from an OpSet is equal to the outcome of applying the operations in their serial order.
It is therefore sufficient for us to define the semantics of each operation under serial execution, and we know that this definition will also define its semantics in arbitrary concurrent executions: any state in a concurrent execution corresponds to some OpSet, and the operations in that OpSet are serializable.

\section{Related Work}\label{sec:relwork}

\subsection{Interpretation of Operation Sequences}\label{sec:op-sequences}

The general idea of establishing a total order of operations, and executing them in that order, appears in many areas of computing:
for example, in the state machine approach to replication \cite{Schneider:1990vy},
the event sourcing approach to data modelling \cite{Vernon:2013ww},
write-ahead logs for crash recovery \cite{Mohan:1992fe},
serializable transactions \cite{Davidson:1985hv},
and scalable multicore data structures \cite{BoydWickizer:2014uz}.
However, beneath the superficial similarity of these approaches there are important differences that need to be distinguished.

As discussed in \S~\ref{sec:order-change}, many of these systems rely on the property that after some operation is executed, all subsequent operations will appear \emph{after} it in the total order.
In other words, the operation sequence is an append-only log, and new operations never need to be inserted ahead of an existing operation in the total order.
This is a very strong property: in the context of a distributed system, it requires an atomic broadcast (or total order broadcast) protocol \cite{Defago:2004ji}, which is equivalent to solving distributed consensus \cite{Chandra:1996cp}.
This class of protocols requires communication with a quorum of nodes in order to make progress \cite{Howard:2016tz}, and it cannot guarantee progress in a fully asynchronous setting \cite{Fischer:1985tt}.

By contrast, the sequential OpSet interpretation of \S~\ref{sec:op-serial} does not require atomic broadcast because it allows operations to be added to the OpSet in any order, and it assigns operation IDs without any coordination.
Few systems use this approach; the most closely related prior work are the Bayou system \cite{Terry:1995dn}, which executes tentative transactions deterministically in timestamp order, and Burckhardt's \emph{standard conflict resolution} \cite[\S~4.3.3]{Burckhardt:2014hy}.
Both of these share the OpSet approach's characteristic that operations with a higher ID need to be undone and re-applied when a new operation with a lower ID is received.

Our contribution in this paper is to formulate the OpSet approach more generally as a tool for specifying and reasoning about complex replicated data structures, such as lists and trees.
Our work is the first to use this approach in mechanised proofs, in which we show that a non-OpSet list CRDT (RGA) satisfies an OpSet-based specification, and prove the absence of the interleaving anomaly in Figure~\ref{fig:bad-merge}.

Baquero et al.~\cite{Baquero:2014ed} and Grishchenko~\cite{Grishchenko:2014eh} have proposed representing CRDTs in terms of a partially-ordered log of operations, where the partial order captures the causal relationships between operations.
The OpSet approach can be seen as a variant of this idea, in which we define the total order on identifiers to be a linear extension of the partial order.

\subsection{Specification and Verification of Replicated Datatypes}

Algorithms for collaboratively editing a shared data structure have been the topic of active research for approximately 30 years, under the headings of Operational Transformation \cite{Ellis:1989ue,Ressel:1996wx,Sun:1998vf,Oster:2006tr} and CRDTs \cite{Shapiro:2011wy,Shapiro:2011un}.
However, throughout this time, the exact consistency properties provided by the algorithms have been somewhat unclear.
For example, Sun et al.~\cite{Sun:1998un} identified three desirable properties that they articulated informally: \emph{convergence}, \emph{causality preservation}, and \emph{intention preservation}.
While the definition of the first two properties is fairly unambiguous, the definition of ``intention preservation'' leaves much more room for interpretation.
Efforts to formally specify and verify the semantics of replicated datatypes have replaced such informal statements with precise consistency properties.

Burckhardt et al.~\cite{Burckhardt:2014ft} provide a wide-ranging formal account of CRDTs, covering their specification, verification, and optimality, with the semantics of an operation on a replicated datatype given as a function of the operation, $o$, and a \emph{operation context}---the set of operations visible to a node at the time that $o$ was received.
Our OpSets can be seen as an explicitly executable variation on this idea: nodes record all operations that they have ever received in a monotonically growing set, and the interpretation function builds the result ``bottom up'' in a fold-like operation.
In contrast to Burckhardt et al., who focus on applying their techniques to set and counter datatypes, we apply our approach to the specification of lists, maps, and trees, using our OpSets as a tool for designing new replicated datatypes---including those previously thought impossible, such as our replicated tree with atomic move.
Gotsman et al.~\cite{DBLP:conf/popl/GotsmanYFNS16} extend Burckhardt et al.'s formalism to reason about hybrid consistency models, providing a modular proof rule inspired by permissions-based logics to enforce an integrity invariant for a given consistency model.

Bieniusa et al.~\cite{Bieniusa:2012gt} articulate a \emph{principle of permutation equivalence} that partially specifies the expected semantics of replicated datatypes, but which leaves some combinations of operations unspecified.
Zeller et al.~\cite{Zeller:2014fl} formalise counters, registers, and sets using Isabelle/HOL and provide mechanised proofs of their correctness.
Attiya et al.~\cite{Attiya:2016kh} give two specifications of collaborative text editing ($\mathcal{A}_\textsf{strong}$ and $\mathcal{A}_\textsf{weak}$), prove that the RGA CRDT \cite{Roh:2011dw} satisfies $\mathcal{A}_\textsf{strong}$, and conjecture that the Operational Transformation algorithm Jupiter \cite{Nichols:1995fd} satisfies $\mathcal{A}_\textsf{weak}$.
Wei et al.~\cite{Wei:2017tg} complete the proof that Jupiter satisfies $\mathcal{A}_\textsf{weak}$.

In our prior work \cite{Gomes:2017gy} we establish a formal verification framework for CRDTs in Isabelle/HOL, and verify the strong eventual consistency properties (in particular, convergence) of a list, set, and counter datatype.
The Isabelle implementation of RGA we use in \S~\ref{sec:datatypes} is based on this work \cite{Gomes:2017vo}.
However, this work does not specify the datatype semantics beyond the convergence property.

Gaducci et al.~\cite{DBLP:conf/coordination/GadducciMR17} develop a semantics for replicated datatypes, placing a focus on compositionality, where a replicated datatype is modelled as a function from labelled directed acyclic graphs of events to sets of values, with each value in this set potentially observable at a node under different ordering of events observed at that node.
A notion of behavioural \emph{refinement} for replicated datatypes induced by set inclusion is also defined, along with a generalisation of their relational semantics to a categorical one.

Mukund et al.~\cite{DBLP:conf/atva/MukundRS15} use traces to provide bounded declarative specifications of CRDTs and show how Counter Example Guided Abstract Refinement (CEGAR) can be used to automatically verify a reference CRDT implementation against its bounded specification.

\subsection{Collaborative Tree Datatypes}

For collaborative editing of tree data structures, several CRDTs \cite{Martin:2010ih,Kleppmann:2016ve} and Operational Transformation algorithms \cite{Jungnickel:2016cb,Ignat:2003jy,Davis:2002iv} have been proposed.
However, most of them only consider insertion and deletion of tree nodes, but do not support a move operation.

As explained in \S~\ref{sec:tree}, supporting an operation that can move a subtree to a new location within a tree introduces new conflicts that need to be handled.
Ahmed-Nacer et al.~\cite{AhmedNacer:2012us} survey approaches to handling these conflicts without providing concrete algorithms.
Tao et al.~\cite{Tao:2015gd} propose handling conflicting move operations by allowing the same object to appear in more than one location; thus, their datatype is strictly a DAG, not a tree.

Najafzadeh~\cite{Najafzadeh:2017vk,Najafzadeh:2018bw} asserts that concurrent move operations on a tree cannot safely be implemented in a CRDT, since the precondition of a move operation is not stable.
Najafzadeh suggests the use of locks to globally synchronise move operations, preventing a scenario such as that in Figure~\ref{fig:concurrent-move} from ever occurring.
However, the resulting datatype is not strictly a CRDT, since some operations require strongly consistent synchronisation.

To our knowledge, our move semantics specified in \S~\ref{sec:tree} is the first definition of such an operation on a fully asynchronous tree CRDT.
We avoid the apparent contradiction with Najafzadeh's assertion by evaluating the precondition $(\mathit{val},\, \mathit{obj}) \notin \mathsf{ancestor}(E)$ at the same time as applying the operation, rather than at the time when the operation is generated, and by applying all operations in the OpSet in a deterministic order.


% Topological orderings are also closely related to the concept of a linear
% extension of a partial order in mathematics.
% https://en.wikipedia.org/wiki/Topological_sorting

\begin{figure*}
\begin{align*}
    \mathrm{listElem}(\mathit{ID}) \leftarrow\; &
    \mathrm{insert}(\mathit{Parent}, \mathit{ID}, \mathit{Value}).
\\
    \mathrm{hasChild}(\mathit{Parent}) \leftarrow\; &
    \mathrm{insert}(\mathit{Parent}, \mathit{Child}, \mathit{Value}).
\\
    \mathrm{child}(\mathit{Parent}, \mathit{Child}) \leftarrow\; &
    \mathrm{insert}(\mathit{Parent}, \mathit{Child}, \mathit{Value}).
\\
    \mathrm{laterChild}(\mathit{Parent}, \mathit{Child}) \leftarrow\; &
    \mathrm{child}(\mathit{Parent}, \mathit{Prev}),
    \mathrm{child}(\mathit{Parent}, \mathit{Child}),
    \mathit{Child} < \mathit{Prev}.
\\
    \mathrm{firstChild}(\mathit{Parent}, \mathit{Child}) \leftarrow\; &
    \mathrm{child}(\mathit{Parent}, \mathit{Child}),
    \neg\mathrm{laterChild}(\mathit{Parent}, \mathit{Child}).
\\
    \mathrm{sibling}(\mathit{Child1}, \mathit{Child2}) \leftarrow\; &
    \mathrm{child}(\mathit{Parent}, \mathit{Child1}),
    \mathrm{child}(\mathit{Parent}, \mathit{Child2}).
\\
    \mathrm{laterSibling}(\mathit{Prev}, \mathit{Later}) \leftarrow\; &
    \mathrm{sibling}(\mathit{Prev}, \mathit{Later}),
    \mathit{Later} < \mathit{Prev}.
\\
    \mathrm{later2Sibling}(\mathit{Prev}, \mathit{Later}) \leftarrow\; &
    \mathrm{sibling}(\mathit{Prev}, \mathit{Next}),
    \mathrm{sibling}(\mathit{Prev}, \mathit{Later}),
    \mathit{Later} < \mathit{Next},
    \mathit{Next} < \mathit{Prev}.
\\
    \mathrm{nextSibling}(\mathit{Prev}, \mathit{Next}) \leftarrow\; &
    \mathrm{laterSibling}(\mathit{Prev}, \mathit{Next}),
    \neg\mathrm{later2Sibling}(\mathit{Prev}, \mathit{Next}).
\\
    \mathrm{hasNextSibling}(\mathit{Prev}) \leftarrow\; &
    \mathrm{laterSibling}(\mathit{Prev}, \mathit{Next}).
\\
    \mathrm{nextSiblingAnc}(\mathit{Start}, \mathit{Next}) \leftarrow\; &
    \mathrm{nextSibling}(\mathit{Start}, \mathit{Next}).
\\
    \mathrm{nextSiblingAnc}(\mathit{Start}, \mathit{Next}) \leftarrow\; &
    \neg\mathrm{hasNextSibling}(\mathit{Start}),
    \mathrm{child}(\mathit{Parent}, \mathit{Start}),
    \mathrm{nextSiblingAnc}(\mathit{Parent}, \mathit{Next}).
\\
    \mathrm{nextElem}(\mathit{Prev}, \mathit{Next}) \leftarrow\; &
    \mathrm{firstChild}(\mathit{Prev}, \mathit{Next}).
\\
    \mathrm{nextElem}(\mathit{Prev}, \mathit{Next}) \leftarrow\; &
    \mathrm{listElem}(\mathit{Prev}),
    \neg\mathrm{hasChild}(\mathit{Prev}),
    \mathrm{nextSiblingAnc}(\mathit{Prev}, \mathit{Anc}, \mathit{Next}).
\end{align*}
\caption{Datalog rules for an ordered list (insertion only).}
\end{figure*}

{\footnotesize
\bibliographystyle{plainnat}
\bibliography{references}{}}
\end{document}
