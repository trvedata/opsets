\documentclass[twocolumn,10pt]{article}
\usepackage[a4paper,margin=2cm]{geometry}
\usepackage[utf8]{inputenc}
\usepackage{mathptmx} % times roman, including math
\usepackage[hyphens]{url}
\usepackage{doi}
\usepackage{hyperref}
\usepackage[numbers,sort]{natbib}
\usepackage{amsmath}
\hyphenation{da-ta-cen-ter da-ta-cen-ters}
\frenchspacing

\begin{document}
\sloppy
\title{OpSets: Concurrent Datatypes with Sequential Specifications}
\author{}
\date{}
\maketitle

\subsection*{Abstract}
TODO

\section{Introduction}

A common requirement across a great variety of systems is that several participants may concurrently access and manipulate some shared data structure.
However, unconstrained concurrent manipulation may introduce inconsistencies that applications cannot tolerate.
Thus, various approaches for providing \emph{consistency guarantees} have been developed, ensuring that the data structure continues to obey certain invariants or semantics under concurrent access. For example:

\begin{description}
\item[Transaction isolation] in databases restricts the degree to which concurrent transactions can affect each other while accessing the same database.
The strongest isolation level, \emph{serializability}, ensures that transactions behave as if there were no concurrency at all, i.e.\ as if transactions were executed serially, one at a time.

\item[Conflict-free Replicated Data Types (CRDTs)] allow each participant to modify a local copy (\emph{replica}) of a shared data structure without waiting for communication with other replicas.
This has the consequence that the state of replicas may temporarily diverge, but the definition of CRDTs ensures that all replicas eventually converge towards a consistent state.

\item[Operational Transformation (OT)] algorithms are designed for several users who are collaboratively editing a shared document, as implemented for example in Google Docs.
As with CRDTs, OT allows each user's changes to be applied immediately to their local copy, while ensuring that other users' concurrent edits can be integrated in a consistent way.
\end{description}

We discuss these techniques in more detail in Section~\ref{sec:relwork}.
Despite decades of research in the above topics, consistency mechanisms for concurrent data access remain poorly understood, and tools for reasoning about them (both formally and informally) are subtle and error-prone.

In this work we introduce \emph{Operation Sets} (or \emph{OpSets} for short), a novel approach for describing and reasoning about algorithms for concurrent data access and manipulation.
The OpSets approach trivially ensures that replicas converge, and it simplifies the tasks of extending algorithms and verifying more complex correctness properties.
It builds upon existing work on CRDTs, and extends that work by incorporating ideas from database query languages, transaction isolation, and operational transformation.

Our contributions in this paper are as follows:

\begin{itemize}
\item We demonstrate how to express a variety of abstract datatypes (maps, lists, trees, and registers) as declarative definitions on top of a simple abstraction, the OpSet.
By using the Datalog query language for our definitions, we strike an effective compromise between two competing requirements: being abstract enough to make formal verification feasible, and being concrete enough to enable efficient practical implementations.

\item We define correctness properties of the above abstract datatypes using a generalised notion of serializability, and we use the Isabelle/HOL proof assistant to formally verify that our implementation satisfies those correctness properties.
Furthermore, we demonstrate that our correctness properties capture intuitive requirements by giving counter-examples for several existing algorithms in the literature, showing that they do not satisfy the required semantics.

\item To demonstrate the ease of extending OpSet definitions with new operations, we provide an implementation of a \emph{move} operation for lists and trees~-- an operation that is not supported by prior CRDT algorithms~-- and we verify its correctness against a specification.

\item We show how to perform garbage collection on OpSets to prevent unbounded growth on a dataset. Unlike several prior proposals for garbage collection in CRDTs, our approach does not require any global coordination or consensus between nodes.

\item We discuss the distinction between three families of CRDTs in the literature (operation-based, state-based, and delta-based), and argue that our OpSet approach makes this distinction largely redundant: we require only a basic set CRDT as foundation, and show that other abstract datatypes can efficiently be implemented on top of that set, without making further assumptions about the properties of the underlying network.
\end{itemize}

\section{The OpSets Approach}\label{sec:approach}

The OpSets approach is a simple abstraction for describing the consistency properties of a replicated data system.
We outline the general approach in this section, before describing concrete data structures and specifications in \S~\ref{sec:datatypes} and \S~\ref{sec:tree}.

\subsection{System Model}\label{sec:system-model}

We assume that the system consists of a set of \emph{nodes} connected by a \emph{network}.
These nodes concurrently access some \emph{shared data structure}, which may be a relational database (consisting of rows in tables), a text document (a sequence of characters), a vector-graphics document (a tree of records describing graphical objects), a filesystem (a tree of directories and files), or any other kind of data structure.

New nodes can be added at any time, and the set of nodes need not be known in advance.
Nodes might be mobile devices, and hence we assume that nodes are sometimes \emph{offline}, i.e.\ temporarily unable to communicate with other nodes.
We require that nodes can access the shared data anytime, even while offline.
Thus, each node has a local copy of the shared data structure, which it can read and modify without waiting for any communication or coordination with other nodes.

Whenever a node makes a modification to that structure, it records the change as an \emph{operation}.
For example, an operation may describe an insertion at a particular position in a text document.
Each node locally maintains a set of operations, the \emph{OpSet}.
Whenever a node makes a change to the shared data, it adds the corresponding operation to its OpSet, and also sends \emph{messages} containing the operation to other nodes.
Whenever a node receives a message from another node, the operation in that message is added to the recipient's local OpSet.
Operations remain immutable throughout this process.

We make no assumptions about the reliability of the network: messages may be lost, duplicated, or arbitrarily reordered.
Reflecting the characteristics of real networks, we assume that lost messages are retransmitted when possible (e.g.\ using TCP), but messages may be permanently lost due to network or node failures.
Since the OpSet at each node is a monotonically growing set of operations, any two communicating nodes can merge their OpSets using the standard set union operator $\cup$.
Set union is commutative, associative, and idempotent, ensuring that communicating nodes converge towards the same OpSet contents.

We assume that each operation has a unique identifier (ID), that new IDs can be generated by any node without communication with other nodes, and that we have a total ordering on operation IDs.
These requirements can easily be met by using Lamport timestamps \cite{Lamport:1978jq} as IDs.
A Lamport timestamp is a pair $(\mathit{counter}, \mathit{nodeID})$ that is constructed as follows:
\begin{itemize}
\item $\mathit{counter}$ is an integer.
    To generate a new ID, find the maximum counter of any existing operation ID in the local OpSet, and increment that number.
\item $\mathit{nodeID}$ is a string that uniquely identifies the node generating the ID, e.g.\ a UUID \cite{Leach:2005hm}.
\end{itemize}

Although different nodes may generate IDs with the same counter value, each node generates IDs with strictly increasing counter values, and thus IDs are globally unique.
We define the total order on IDs as being the lexicographic order:
\[
    \,(\mathit{ctr}_1, \mathit{node}_1) < (\mathit{ctr}_2, \mathit{node}_2)\,
    \;\Longleftrightarrow\;
    \mathit{ctr}_1 < \mathit{ctr}_2 \;\vee\;
    (\mathit{ctr}_1 = \mathit{ctr}_2 \wedge \mathit{node}_1 <\mathit{node}_2).
\]

%This total order is a linear extension of the partial order that captures their causal dependencies (the \emph{happens-before} relation).
%That is, whenever a node changes the shared data and generates a new operation, the ID of the new operation must be strictly greater than that of any existing operation in the OpSet of the node generating the new operation.

\subsection{Interpreting an OpSet}\label{sec:op-serial}

Most definitions of operation-based CRDTs describe how a node's local state is manipulated by operations \cite{Shapiro:2011wy,Shapiro:2011un}.
We now depart from this convention and present an alternative formulation of replicated datatypes.

In the OpSets approach, we require that the shared data structure is never manipulated directly.
Instead, we use an \emph{interpretation function} $\llbracket-\rrbracket$ that takes an OpSet $O$ and returns the current state $\llbracket O \rrbracket$ of the shared data structure described by the OpSet.
The interpretation function is \emph{pure}, i.e.\ deterministic, side-effect free, and its result depends only on $O$.
All nodes in the system employ the same interpretation function.

Consequently, whenever any two nodes have the same OpSet $O$, their view of the shared data structure $\llbracket O \rrbracket$ must also be equal.
This construction trivially ensures eventual consistency: as two nodes converge towards the same OpSet contents, any data structure that is deterministically derived from the OpSet must also converge.

In principle, any deterministic function can serve as interpretation function.
However, in defining the semantics of CRDTs (see \S~\ref{sec:datatypes} and \S~\ref{sec:tree}), we have found it useful to specialise $\llbracket-\rrbracket$ such that we can interpret one operation at a time.

Let the OpSet $O$ be a set of pairs $(\mathit{id},\, \mathit{op})$, where $\mathit{id}$ is a unique operation identifier and $\mathit{op}$ is an arbitrary description of the change that occurred.
Assume that we have a total ordering $<$ on identifiers, as explained in \S~\ref{sec:system-model}.
Then observe that for any OpSet there exists a unique sequence of operations, containing all operations of the OpSet in ascending order of their identifier.
We can specify the semantics of each operation~--- that is, the effect of the operation on the OpSet interpretation~--- when applied in this sequential order.

Formally, we can define the interpretation $\llbracket O \rrbracket$ of the OpSet $O$ as follows:
\begin{align*}
    \big\llbracket \emptyset \big\rrbracket &= \mathsf{InitialState} \\
    \big\llbracket O \;\cup\; \{(\mathit{id},\, \mathit{op})\} \big\rrbracket &=
    \mathsf{interp}\big[\llbracket O \rrbracket,\, (\mathit{id},\, \mathit{op})\big]
    \qquad\text{ provided that } \forall\,(\mathit{id}',\, \mathit{op}') \in O.\; \mathit{id}' < \mathit{id}
\end{align*}
where $\mathsf{interp}\big[S,\, (\mathit{id},\, \mathit{op})\big]$ is the interpretation of the operation $(\mathit{id},\, \mathit{op})$ in the state $S$, and $\mathsf{InitialState}$ is a fixed minimal element (e.g. the empty tree, or empty list) of the replicated type described.
In other words, if $S$ is the result of interpreting all operations with identifiers less than $\mathit{id}$, then
$\mathsf{interp}\big[S,\, (\mathit{id},\, \mathit{op})\big]$ is the interpretation of the OpSet to which $(\mathit{id},\, \mathit{op})$ has been added.
For example, if $\mathit{id}_1 < \mathit{id}_2 < \mathit{id}_3$, we have:
\begin{align*}
    \big\llbracket \{(\mathit{id}_1,\ \mathit{op}_1),\;
    &(\mathit{id}_2,\ \mathit{op}_2),\,
    (\mathit{id}_3,\ \mathit{op}_3)\} \big\rrbracket \;=\\
    &\mathsf{interp}\big[\mathsf{interp}\big[\mathsf{interp}\big[\mathsf{InitialState},\,
    (\mathit{id}_1,\ \mathit{op}_1)\big],\,
    (\mathit{id}_2,\ \mathit{op}_2)\big],\,
    (\mathit{id}_3,\ \mathit{op}_3)\big]
\end{align*}
Provided that the operation interpretation $\mathsf{interp}\big[S,\, (\mathit{id},\, \mathit{op})\big]$ is deterministic, the OpSet interpretation function $\llbracket-\rrbracket$ is also deterministic, due to the fact that the operation order in the OpSet is unique.

\subsection{Receiving Messages Out-of-order}\label{sec:order-change}

Many computing systems are based on the idea of putting operations in some total order, and executing them in that order.
For example, serializable transactions \cite{Kleppmann:2017wj} and state machine replication \cite{Schneider:1990vy} follow this approach.
However, it is important to understand that the OpSet interpretation of \S~\ref{sec:op-serial} relies on a weaker notion of ordering than most systems.

With serializable transactions and state machine replication, once a transaction/operation has been executed in some state, its results are expected to be durable.
Thus, before executing some transaction $T_i$, the system needs to ensure that there is no pending transaction with a lower ID than $T_i$ (which would need to be executed before $T_i$), since otherwise the subsequent arrival of a transaction with lower ID would invalidate the state in which $T_i$ was executed.
However, ensuring this precondition is expensive: as we show in \S~\ref{sec:op-sequences}, it requires communication with at least a quorum of nodes; if the IDs are Lamport timestamps, it even requires communication with every single node \cite{Lamport:1978jq}.
If too many nodes are offline, the system cannot execute any transactions.

By contrast, our system model of \S~\ref{sec:system-model} requires nodes to always be able to read and modify the shared data, even when all nodes are offline.
Moreover, we do not assume any ordering guarantees from the network.
Thus, whenever there is some operation $(\mathit{id}_1, \mathit{op}_1) \in O$ in the OpSet $O$ of some node, it is possible that the node will subsequently receive a message containing $(\mathit{id}_2, \mathit{op}_2)$, where $\mathit{id}_2 < \mathit{id}_1$; that is, the later-arriving operation needs to be applied before the existing operation $(\mathit{id}_1, \mathit{op}_1)$ in the OpSet interpretation $\llbracket O \rrbracket$.

In the OpSet model, such out-of-order delivery of operations is no problem: the order in which operations are received has no effect on the OpSet $O$, and since we assume the interpretation function to be pure and side-effect free, the interpretation $\llbracket O \rrbracket$ can always be recomputed whenever new operations are added to $O$.

The interpretation function is an \emph{executable specification} that defines the expected result of interpreting a set of operations.
Presenting replicated datatypes in this manner has two significant advantages:
\begin{enumerate*}
\item
Unlike typical definitions of CRDT algorithms \cite{Shapiro:2011wy,Shapiro:2011un}, it is not necessary for the interpretation function $\mathsf{interp}\big[S,\, (\mathit{id},\, \mathit{op})\big]$ to commute with respect to other operations: any pure function can be used.
This fact makes it much simpler to specify the interpretation of operations, as we shall see in \S~\ref{sec:datatypes} and \S~\ref{sec:tree}.
\item
We can guarantee the existence of an implementation of each described datatype: the specification itself.
This is in contrast to axiomatic specifications, which may not be implementable, and require additional work to demonstrate than an implementation exists which satisfies the axiomatic description.
\end{enumerate*}

For practical implementations of replicated datatypes, a naive OpSet interpretation may exhibit poor performance, since nodes must potentially apply the same subset of operations repeatedly.
More efficient (and, most likely, more complex) algorithms for CRDTs can therefore be developed and shown to satisfy the OpSet-based specification---we do this in \S~\ref{sec:bad-merge}.

However, we have developed a practical JavaScript CRDT implementation around the OpSet model,\footnote{\url{https://github.com/automerge/automerge}} and found it to have some advantages: for example, users can easily inspect the editing history of a document, since every past version of the document is the interpretation of a particular subset of operations.
Moreover, using OpSets provides a straightforward mechanism for recovering from network partitions and failures, as missing operations may be retransmitted and added to the OpSets of previously partitioned nodes.
The details of this implementation are beyond the scope of this paper.

\section{Related Work}\label{sec:relwork}

Algorithms for collaboratively editing a shared data structure have been the topic of active research for approximately 30 years, under the headings of Operational Transformation \cite{Ellis:1989ue,Nichols:1995fd,Ressel:1996wx,Sun:1998un,Sun:1998vf,Suleiman:1997gl,Suleiman:1998eu,Vidot:2000ch,Imine:2003ks,Li:2004er,Li:2008hw,Oster:2006tr} and CRDTs \cite{Shapiro:2011wy,Shapiro:2011un,Roh:2011dw,Preguica:2009fz,Oster:2006wj,Weiss:2010hx,Nedelec:2013ky,Nedelec:2016eo,Grishchenko:2014eh,Kleppmann:2016ve}.

However, throughout this time, the exact consistency properties provided by the algorithms have been somewhat unclear.
For example, Sun et al.~\cite{Sun:1998un} identified three desirable properties that they articulated informally: \emph{convergence}, \emph{causality preservation}, and \emph{intention preservation}.
While the definition of the first two properties is fairly unambiguous, the definition of ``intention preservation'' leaves much more room for interpretation.
Sun et al.\ define it as follows \cite{Sun:1998un}:
\begin{displayquote}
For any operation $O$, the effects of executing $O$ at all sites are the same as the intention of $O$, and the effect of executing $O$ does not change the effects of independent operations.
\end{displayquote}
where the term \emph{intention} is in turn defined as:
\begin{displayquote}
The intention of an operation $O$ is the execution effect which can be achieved by applying $O$ on the document state from which $O$ was generated.
\end{displayquote}
Efforts to formally specify and verify the semantics of replicated datatypes have replaced informal statements of this type with more precise definitions of consistency properties.

\subsection{Specification and Verification}

Bieniusa et al.~\cite{Bieniusa:2012gt} articulate a \emph{principle of permutation equivalence} that partially specifies the expected semantics of replicated datatypes, but which leaves some combinations of operations unspecified.
Burckhardt et al.~\cite{Burckhardt:2014ft} give a complete specification of CRDT counters, registers, and sets, and show how to verify that algorithms satisfy these specifications in hand-written proofs.
Zeller et al.~\cite{Zeller:2014fl} formalise the same datatypes using Isabelle/HOL and provide mechanised proofs of their correctness.
These papers do not consider lists, maps, or tree datatypes.

Gomes et al.~\cite{Gomes:2017gy} establish a formal verification framework for CRDTs in Isabelle/HOL, and verify the strong eventual consistency properties (in particular, convergence) of a list, set, and counter datatype.
However, the work does not specify the datatype semantics beyond the convergence property.

Attiya et al.~\cite{Attiya:2016kh} give two specifications of collaborative text editing ($\mathcal{A}_\textsf{strong}$ and $\mathcal{A}_\textsf{weak}$), prove that the RGA CRDT \cite{Roh:2011dw} satisfies $\mathcal{A}_\textsf{strong}$, and conjecture that the Operational Transformation algorithm Jupiter \cite{Nichols:1995fd} satisfies $\mathcal{A}_\textsf{weak}$.
Wei et al.~\cite{Wei:2017tg} complete the proof that Jupiter satisfies $\mathcal{A}_\textsf{weak}$.

XXX: also~\cite{DBLP:conf/popl/BurckhardtGYZ14, DBLP:conf/atva/MukundRS15, DBLP:conf/coordination/GadducciMR17, DBLP:conf/popl/GotsmanYFNS16}

\subsection{Collaborative Tree Datatypes}

For collaborative editing of tree data structures, several CRDTs \cite{Martin:2010ih,Kleppmann:2016ve} and Operational Transformation algorithms \cite{Jungnickel:2016cb,Ignat:2003jy,Davis:2002iv} have been proposed.
However, most of them only consider insertion and deletion of tree nodes, but do not support a move operation.

As explained in Section~\ref{sec:tree}, supporting an operation that can move a subtree to a new location within a tree raises the possibility of some particular conflicts that need to be handled.
Ahmed-Nacer et al.~\cite{AhmedNacer:2012us} survey approaches to handling these conflicts without providing concrete algorithms.
Tao et al.~\cite{Tao:2015gd} propose handling conflicting move operations by allowing the same object to appear in more than one location; thus, their datatype is strictly a DAG, not a tree.

Najafzadeh~\cite{Najafzadeh:2017vk} asserts that concurrent move operations on a tree cannot safely be implemented in a CRDT, since the precondition of a move operation is not stable.
The proposed solution in this work is to use locks to globally synchronise move operations, thus preventing a scenario such as that in Figure~\ref{fig:concurrent-move} from ever occurring.
However, the resulting datatype is not strictly a CRDT, since some operations require strongly consistent synchronisation.

To our knowledge, our move semantics specified in Section~\ref{sec:tree} is the first definition of such an operation on a fully asynchronous tree CRDT.
We avoid the apparent contradiction with Najafzadeh's assertion by evaluating the precondition $(\mathit{val},\, \mathit{obj}) \notin \mathrm{ancestor}(E)$ at the same time as applying the operation, rather than at the time when the operation is generated, and by applying all operations in the OpSet in a deterministic order.

\subsection{Ordered Sets of Operations}

Baquero et al.~\cite{Baquero:2014ed} and Grishchenko~\cite{Grishchenko:2014eh} have previously proposed representing CRDTs in terms of a partially-ordered log of operations (where the partial order captures the causal relationships between operations).
Our OpSet is a straightforward variant of this idea, in which we define the total order on identifiers to be a linear extension of the partial order that captures causality.
This linear extension is well-known and goes back to Lamport~\cite{Lamport:1978jq}.

Our approach of sequentially interpreting operations, in order of $\mathit{id}$, is reminiscent of the concept of \emph{serializability} in databases: the data structure obtained by interpreting an OpSet is equal to the outcome of applying the operations in their serial order, even if the execution that produced the OpSet was in fact concurrent.
However, conventional transaction serializability requires synchronous coordination between replicas \cite{Davidson:1985hv}.
We circumvent this limitation, and hence allow nodes to make progress while offline, by allowing the interpretation of an operation to change if another operation with a lower ID is delivered.


\begin{figure*}
\begin{align*}
\mathrm{isListElem}(\mathit{Oid}) \leftarrow\; &
    \mathrm{insertAfter}(\mathit{Oid}, \mathit{Parent}).\\
\mathrm{hasChild}(\mathit{Parent}) \leftarrow\; &
    \mathrm{insertAfter}(\mathit{Child}, \mathit{Parent}).\\
\mathrm{laterChild}(\mathit{Parent}, \mathit{Child2}) \leftarrow\; &
    \mathrm{insertAfter}(\mathit{Child1}, \mathit{Parent}),\;
    \mathrm{insertAfter}(\mathit{Child2}, \mathit{Parent}),\;
    \mathit{Child1} > \mathit{Child2}.\\
\mathrm{firstChild}(\mathit{Parent}, \mathit{Child}) \leftarrow\; &
    \mathrm{insertAfter}(\mathit{Child}, \mathit{Parent}),\;
    \neg\;\mathrm{laterChild}(\mathit{Parent}, \mathit{Child}).\\
\mathrm{sibling}(\mathit{Child1}, \mathit{Child2}) \leftarrow\; &
    \mathrm{insertAfter}(\mathit{Child1}, \mathit{Parent}),\;
    \mathrm{insertAfter}(\mathit{Child2}, \mathit{Parent}).\\
\mathrm{laterSibling}(\mathit{Sib1}, \mathit{Sib2}) \leftarrow\; &
    \mathrm{sibling}(\mathit{Sib1}, \mathit{Sib2}),\;
    \mathit{Sib1} > \mathit{Sib2}.\\
\mathrm{laterSibling2}(\mathit{Sib1}, \mathit{Sib3}) \leftarrow\; &
    \mathrm{sibling}(\mathit{Sib1}, \mathit{Sib2}),\;
    \mathrm{sibling}(\mathit{Sib1}, \mathit{Sib3}),\;
    \mathit{Sib1} > \mathit{Sib2},\;
    \mathit{Sib2} > \mathit{Sib3}.\\
\mathrm{nextSibling}(\mathit{Sib1}, \mathit{Sib2}) \leftarrow\; &
    \mathrm{laterSibling}(\mathit{Sib1}, \mathit{Sib2}),\;
    \neg\;\mathrm{laterSibling2}(\mathit{Sib1}, \mathit{Sib2}).\\
\mathrm{hasNextSibling}(\mathit{Sib1}) \leftarrow\; &
    \mathrm{laterSibling}(\mathit{Sib1}, \mathit{Sib2}).\\
\mathrm{nextSiblingAnc}(\mathit{Start}, \mathit{Next}) \leftarrow\; &
    \mathrm{nextSibling}(\mathit{Start}, \mathit{Next}).\\
\mathrm{nextSiblingAnc}(\mathit{Start}, \mathit{Next}) \leftarrow\; &
    \neg\;\mathrm{hasNextSibling}(\mathit{Start}),\;
    \mathrm{insertAfter}(\mathit{Start}, \mathit{Parent}),\;
    \mathrm{nextSiblingAnc}(\mathit{Parent}, \mathit{Next}).\\
\mathrm{hasSiblingAnc}(\mathit{Start}) \leftarrow\; &
    \mathrm{nextSiblingAnc}(\mathit{Start}, \mathit{Next}).\\
\mathrm{nextElem}(\mathit{Prev}, \mathit{Next}) \leftarrow\; &
    \mathrm{firstChild}(\mathit{Prev}, \mathit{Next}).\\
\mathrm{nextElem}(\mathit{Prev}, \mathit{Next}) \leftarrow\; &
    \mathrm{isListElem}(\mathit{Prev}),\;
    \neg\;\mathrm{hasChild}(\mathit{Prev}),\;
    \mathrm{nextSiblingAnc}(\mathit{Prev}, \mathit{Next}).\\

\end{align*}
\caption{Datalog rules for an ordered list (insertion only).}
\end{figure*}

{\footnotesize
\bibliographystyle{plainnat}
\bibliography{references}{}}
\end{document}
