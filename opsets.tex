\documentclass[twocolumn,10pt]{article}
\usepackage[a4paper,margin=2cm]{geometry}
\usepackage[utf8]{inputenc}
\usepackage{mathptmx} % times roman, including math
\usepackage[hyphens]{url}
\usepackage{doi}
\usepackage{hyperref}
\usepackage[numbers,sort]{natbib}
\usepackage{amsmath}
\hyphenation{da-ta-cen-ter da-ta-cen-ters}
\frenchspacing

\begin{document}
\sloppy
\title{OpSets}
\author{}
\date{}
\maketitle

\subsection*{Abstract}

\section{Introduction}

\cite{Attiya:2016kh}

% Topological orderings are also closely related to the concept of a linear
% extension of a partial order in mathematics.
% https://en.wikipedia.org/wiki/Topological_sorting

\begin{figure*}
\begin{align*}
    \mathrm{listElem}(\mathit{ID}) \leftarrow\; &
    \mathrm{insert}(\mathit{Parent}, \mathit{ID}, \mathit{Value}).
\\
    \mathrm{hasChild}(\mathit{Parent}) \leftarrow\; &
    \mathrm{insert}(\mathit{Parent}, \mathit{Child}, \mathit{Value}).
\\
    \mathrm{child}(\mathit{Parent}, \mathit{Child}) \leftarrow\; &
    \mathrm{insert}(\mathit{Parent}, \mathit{Child}, \mathit{Value}).
\\
    \mathrm{laterChild}(\mathit{Parent}, \mathit{Child}) \leftarrow\; &
    \mathrm{child}(\mathit{Parent}, \mathit{Prev}),
    \mathrm{child}(\mathit{Parent}, \mathit{Child}),
    \mathit{Child} < \mathit{Prev}.
\\
    \mathrm{firstChild}(\mathit{Parent}, \mathit{Child}) \leftarrow\; &
    \mathrm{child}(\mathit{Parent}, \mathit{Child}),
    \neg\mathrm{laterChild}(\mathit{Parent}, \mathit{Child}).
\\
    \mathrm{sibling}(\mathit{Child1}, \mathit{Child2}) \leftarrow\; &
    \mathrm{child}(\mathit{Parent}, \mathit{Child1}),
    \mathrm{child}(\mathit{Parent}, \mathit{Child2}).
\\
    \mathrm{laterSibling}(\mathit{Prev}, \mathit{Later}) \leftarrow\; &
    \mathrm{sibling}(\mathit{Prev}, \mathit{Later}),
    \mathit{Later} < \mathit{Prev}.
\\
    \mathrm{later2Sibling}(\mathit{Prev}, \mathit{Later}) \leftarrow\; &
    \mathrm{sibling}(\mathit{Prev}, \mathit{Next}),
    \mathrm{sibling}(\mathit{Prev}, \mathit{Later}),
    \mathit{Later} < \mathit{Next},
    \mathit{Next} < \mathit{Prev}.
\\
    \mathrm{nextSibling}(\mathit{Prev}, \mathit{Next}) \leftarrow\; &
    \mathrm{laterSibling}(\mathit{Prev}, \mathit{Next}),
    \neg\mathrm{later2Sibling}(\mathit{Prev}, \mathit{Next}).
\\
    \mathrm{hasNextSibling}(\mathit{Prev}) \leftarrow\; &
    \mathrm{laterSibling}(\mathit{Prev}, \mathit{Next}).
\\
    \mathrm{nextSiblingAnc}(\mathit{Start}, \mathit{Start}, \mathit{Next}) \leftarrow\; &
    \mathrm{nextSibling}(\mathit{Start}, \mathit{Next}).
\\
    \mathrm{nextSiblingAnc}(\mathit{Start}, \mathit{Anc}, \mathit{Next}) \leftarrow\; &
    \neg\mathrm{hasNextSibling}(\mathit{Start}),
    \mathrm{child}(\mathit{Parent}, \mathit{Start}),
    \mathrm{nextSiblingAnc}(\mathit{Parent}, \mathit{Anc}, \mathit{Next}).
\\
    \mathrm{nextElem}(\mathit{Prev}, \mathit{Next}) \leftarrow\; &
    \mathrm{firstChild}(\mathit{Prev}, \mathit{Next}).
\\
    \mathrm{nextElem}(\mathit{Prev}, \mathit{Next}) \leftarrow\; &
    \mathrm{listElem}(\mathit{Prev}),
    \neg\mathrm{hasChild}(\mathit{Prev}),
    \mathrm{nextSiblingAnc}(\mathit{Prev}, \mathit{Anc}, \mathit{Next}).
\end{align*}
\caption{Datalog rules for an ordered list (insertion only).}
\end{figure*}

{\footnotesize
\bibliographystyle{plainnat}
\bibliography{references}{}}
\end{document}
